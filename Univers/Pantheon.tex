
\chapter{Le panthéon}

Le peuple de l'Empire et d'ailleurs ont leur croyances et leurs us. Voici un bref aperçu des divités du jeu. Le panthéon impérial est séparé en deux: 

D'un côté les Aedra, dieux créateurs de Nirn, bienveillants en général envers Tamriel, mais capricieux et très très âgés. Leurs pouvoirs sont en déclin.

De l'autre les Daedra, agents perturbateurs, pour qui Nirn est au choix un jouet, un domaine d'expérimentations, un lieu de vacances, un endroit à conquérir, ou encore une aberration des Aedra qui doit disparaître.

Les autres peuples ont leurs croyances, mais la plupart se basent sur des incarnations des membres de ce Panthéon.

\section{Aedra}

\begin{description}
\item[Akatosh]
{
  Dieu du temps et de la stabilité.
}
\item[Dibella]
{
  Déesse de la beauté.
}
\item[Arkay]
{
  Dieu de la vie et de la mort, du cycle de la vie.
}
\item[Zenithar]
{
  Dieu du commerce et des échanges, dieu des relations.
}
\item[Mara]
{
  Déesse de l'amour et de la fertilité.
}
\item[Stendarr]
{
  Dieu de la pitié et de la compassion.
}
\item[Kynareth]
{
  Déesse des éléments, du ciel et de la terre, mais aussi de la Chance. C'est la sainte patronne des marins et des voyageurs.
}
\item[Julianos]
{
  Dieu de la sagesse et de la raison, de la littérature et de la loi.
}
\item[Talos]
{
  L'humain héros devenu dieu. Il est le dieu Humain. Selon le culte des neuf, chaque Empereur développe au cours de son règne la puissance de Talos, et quand il meurt, il redevient une partie de celui-ci. Talos est le plus vénéré des dieux devant Kynareth et Mara.
}
\end{description}

\section{Daedra}

Les Daedra se séparent en deux catégories : les princes, équivalents des Aedra en puissance, et tous les inférieurs, dont nous ne parlerons pas ici.

\begin{description}
\item[Azura]
{
  Princesse de la lune. C'est la seule à pouvoir être considérée comme bienfaisante selon les critères humains parmi tous les princes Daedra. Le plan d'Oblivion qu'elle a créé dans l'Aurbis s'appelle Clair de Lune.
}
\item[Boethia]
{
  Prince du complot et de la traîtrise. On suppose que Jagar Tharn a été aidé par Boethia.
}
\item[Clavicus Vile]
{
  Prince de la soif de pouvoir. On suppose qu'il a aussi aidé Jagar Tharn dans son complot. Il semble capable de s'incarner dans Nirn pour une durée limitée sous la forme d'un gros chien noir.
}
\item[Hermaeus Mora]
{
  Prince de la fatalité et de la mémoire. Le plan d'Oblivion qu'il domine s'appelle Apocryphe.
}
\item[Hircine]
{
  Prince de la chasse, du jeu. Il est responsable de la majorité des maladies transformant des humains en bêtes, telle la lycanthropie, y voyant une façon de s'amuser.
}
\item[Jyggalag]
{
  Prince de l'Ordre. Il a des tas de Daedra mineurs qui le suivent et qui forment les chevaliers de l'Ordre. On sait aujourd'hui que Sheogorath et Jyggalag ne faisaient qu'un.
}
\item[Malacath]
{
  Prince de la rebellion et de la révolte. Il s'agit de l'Aedra Trinimac transformé en Daedra après avoir été absorbé par Boethia.
}
\item[Mehrunes Dagon]
{
  Prince de la destruction, de la colère, du changement et de l'ambition. Le concernant, on est sûr que lui au moins a aidé Jagar Tharn dans son complot.
}
\item[Mephala]
{
  Prince(sse) de la luxure, du meurtre et des secrets. Ce Daedra est le seul dont le sexe change en fonction de son envie. Il apparait comme une femme ou un homme à quatre bras, d'une beauté et d'un charme incroyables, aux courbes ou aux muscles presque parfaits - presque, car cet être sait que la perfection peut lasser. Son jeu favori est de tourner les êtres vivants les uns contre les autres, provoquant ainsi des meurtres. Son deuxième jeu favori est de susurrer à l'oreille des gens perturbés des mots doux, convainquant les personnes en question de commettre viols et orgies...
}
\item[Meridia]
{
  Vivant dans un plan d'Oblivion appelé "Chambres colorées", cette princesse du soleil est peu connue. Elle ne semble pas s'intéresser à Nirn, et on en sait très peu sur elle hormis qu'elle a une immense aversion envers les morts-vivants.
}
\item[Molag Bal]
{
  Prince de la domination et de l'esclavage. Il vit dans un plan d'Oblivion appelé Coldharbour.
}
\item[Namira]
{
  Princesse des mauvais esprits, souvent associée à tous les animaux qui dégoûtent les humains comme les insectes, les vers ...
}
\item[Nocturnal]
{
  Princesse de la nuit et de la discrétion. Elle est vénérée par beaucoup de voleurs.
}
\item[Peryite]
{
  Prince de la pestilence.
}
\item[Sanguine]
{
  Prince de la douleur et du masochisme. Il va souvent de pair avec Mephala sur les portes de beaucoup de maisons closes et de bordels.
}
\item[Sheogorath]
{
  Prince du chaos et de la folie. Son plaisir est de semer le chaos dans le monde d'Ordre qu'est le plan d'Oblivion créé par Jyggalag. On sait aujourd'hui que lui et Jyggalag sont deux personnalités opposées du même Daedra.
}
\item[Vaermina]
{
  Princesse des rêves et des cauchemars. On suppose que les horribles créatures rêvées par les humains apparaissent dans son plan d'Oblivion, Quagmire , et que c'est ainsi que certaines arrivent sur Nirn.
}
\end{description}
