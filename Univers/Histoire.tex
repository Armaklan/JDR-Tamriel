\chapter{Histoire de Nirn}

Voici un historique de Nirn, depuis sa création. Libre à vous de jouer dans la période qui vous semble la plus appropriée, là où vous vous sentirez le plus à l'aise pour créer des intrigues. Cette partie peut sembler touffue pour les novices de l'univers. Aussi, si vous ne connaissez rien à la saga, nous vous conseillons de passer ce chapitre et de jouer au jeu. Le Maître du Jeu vous guidera dans vos premiers pas sur Tamriel !

Le temps est compté depuis le départ en années et en ères, une ère étant l'espace entre deux époques complètement différentes. Entre parenthèses est indiqué l'équivalent en années négatives depuis le début de la 4ème ère.

\section{Ère de l'aube}

Cette époque est si lointaine que l'on a aucune idée des années écoulées depuis ces évènements. Sûrement une dizaine de milliers, davantage encore ...

\begin{description}
\item[Big bang]
Akatosh apparaît dans l'Aurbis et affirme son existence. Aucun univers n'est créé, et Akatosh se sent seul. Il crée des êtres, un peu plus faibles que lui mais considérés aujourd'hui comme les plus puissantes divinités. Tous ont un point commun : soit leur forme est de type Anou (pure énergie) soit Pandoméi (éthérée affectant le monde réel).
\item[Nirn est créée]
Un être pandoméique nommé Lorkhan décide de créer un nouveau type d'élément dans l'Aurbis, hormis les Anou et les Pandoméi : une sphère contenant une dimension, cette dimension contenant une planète appelée Nirn, seule dans le plan d'existence où elle se trouvait.
\item[Aedra et Daedra]
Lorkhan, en créant Nirn, crée un conflit parmi ces êtres divins. Une partie souhaite l'aider à concevoir et à créer de la vie sur Nirn, notamment Magnus, qui sera un des grands architectes de la Création. D'autres trouvent cela ennuyeux, et préfèrent se concentrer sur eux-mêmes et créer un monde à leur image, pour y vivre et y créer sans avoir à discuter avec d'autres de ce qu'ils y mettraient. Aedra et Daedra étaient nés.
\item[La Création]
Notre dimension est crée, appelée Mundus, une sphère dans l'Aurbis. Aujourd'hui, la plupart des divinités qui ont créé historiquement notre monde ne sont plus vénérées dans les différentes mythologies, car ayant abandonné leurs pouvoirs incroyables pour vivre avec les premiers primitifs de ce monde. Ils existent encore comme des ombres de ce qu'ils étaient, cherchant à regagner leur puissance sans que cela soit possible. Pourquoi ont-ils perdu leurs pouvoirs en entrant dans Nirn ? Car Nirn était encore un endroit chaotique, le temps avançait, reculait, s'arrêtait, à sa guise, et la Création devenait autonome, cherchant à sculpter tout ce qu'elle contenait par des éléments plus petits, eux-mêmes composés d'éléments plus petits. Cela ne comprenait pas les être immortels "convertis" alors avec les mêmes composants et éléments que les êtres primitifs. Les mortels apparurent de cette manière, et leurs mythologies communes concernant la création du monde datent de cette époque-là.
\item[Discorde]
Magnus, resté à l'extérieur de la sphère du Mundus, se rendit vite compte de ce qui se passait, du piège mortel qu'était devenu ce Mundus créé par Lorkhan, et abandonna la Création. Mais où aller ? Que faire ? Magnus et les dieux qui le suivaient décidèrent alors de donner à Nirn, afin que celle-ci puisse accueillir des Aedra, le support et la présence de la magie. Cependant Lorkhan n'était pas d'accord, et après que Nirn fut dotée du soleil, de la lune et des étoiles, ainsi que de la magie, ils fuirent pour éviter sa colère.
\item[Normalité]
Akatosh à son tour vit ce que le Mundus était en réalité, mais ne fuit pas Lorkhan. Il alla même dans le Mundus. Il atterrit dans ce qui s'appelle aujourd'hui l'île de Balfiera, en HauteRoche. Il y construit une tour d'Adamantine, qui existe toujours et est la plus vieille bâtisse encore debout. Il utilisa ses pouvoirs, avant qu'ils disparaissent, pour stopper la Création et ses évolutions chaotiques. Le temps devint linéaire et régulier. C'était la fin de l'ère de l'aube. Akatosh appela à un conseil les autres Aedra, qui désormais pouvaient venir sans risques de perdre leurs pouvoirs, afin de voter le sort de Lorkhan, qui fut condamné à mort. Trinimac, un jeune Aedra, arracha le cœur de la poitrine de Lorkhan, et le jeta vers l'Est. Le coeur atterrit tel un monstrueux météore, formant l'île de Vvarfendell et la montagne rouge - ce qui explique que l'île soit appelée Vvarfendell qui signifie "Est blessé par une étoile".
\item[Mortels]
La Création n'était en fait que ralentie, mais elle n'était plus dangereuse pour les dieux, qui y conservaient tous leurs pouvoirs. Le développement de la vie mortelle se continuait, de manière lente, régulière et prévisible. Les peuples existant alors évoluèrent jusqu'à développer leurs langues, leurs connaissances. Le peuple de Tamriel fut appelé "Mer" (Mers). Il y avait les Dwemers (Les Mers sages), les Chimers (les Mers changés), les Bosmers (les Mers de la nature), et les Altmers (les Mers cultivés/anciens). Les peuples humains du continent Atmora aujourd'hui disparu incluaient les Nedes (ancêtres des Impériaux), les Nordiques, et des aborigènes aujourd'hui disparus. Les Ehlnofey, sur Tamriel, étaient des êtres pensants devenus aujourd'hui végétaux (Hist). A Yokuda existaient déjà les rougegardes.
\end{description}

\section{Ère méréthique}

Ce fut une ère de héros. Peu de dates, mais l'on sait que pendant cette période, les peuples Mers découvraient comment communiquer avec les Aedra et les Daedra. Alors que les Altmer conversaient principalement avec leurs créateurs, les Aedra, les Dwemer se consacraient à la science et à la création d'artefacts et d'objets fonctionnant aussi bien à la vapeur qu'avec de la magicka (nom donné à la magie). Les Chimer, eux, étaient dynamiques, sans cesse en quête d'astuce, d'organisation, de stratégie d'action. Les groupes qui allaient devenir les maisons nobles débutèrent à cette époque.

Mais il y eut deux événements majeurs.

Le premier est que les peuples Nedes et Nordique découvrirent une brèche dans le Mundus, par laquelle des Daedra pouvaient passer malgré le blocage des Aedra. Les Aedra firent alors une guerre terrible pour repousser les Daedra, dont le but était ouvertement de conquérir ce monde finalement pas si ennuyeux à leurs yeux par rapport à leurs propres créations. Les peuples humains fuirent alors ce continent, Atmora, pour aller vers Tamriel. Boreath, un des princes Daedra présents dans le conflit, mangea Trinimac, un jeune Aedra. Il le digéra, mais cela ne tua pas l'Aedra, cela le corrompit, il se fit appeler Malacath, et son peuple fut à son tour maudit par les Aedra. Ainsi naquirent les Orsimers.

Le deuxième événement découle du premier. Les humains arrivèrent en Tamriel. Ils découvrirent les Altmer, les Orsimers, les Dunmer et les Ayléides (des Mers sauvages et solitaires). Les Direnni, une puissante famille Altmer, firent construire des tours de dimensions démesurées : La tour de Cristal domina l'archipel de Summerset, la tour d'Or Blanc domina ce qui allait devenir Cyrodiil. Les humains oublièrent leurs expériences avec la magie daédrique, et découvrirent la magie des Altmer, utilisant un composé de la nature, la magicka. Mais les Ayléides trompèrent la confiance des humains, ils vénéraient les Daedra et ce fut vite un fait : les humains étaient devenus des esclaves. Les Altmer, eux, ne bougèrent pas le petit doigt.

\section{1ère ère (-4250 / -1330)}

Plusieurs évènements marquants :

\begin{itemize}
\setlength{\leftmargin}{35pt}
\setlength{\itemsep}{20pt}
\item
1E 1 - Eplear fonde la dynastie Camoran en Valenwood et décide de créer un royaume pour les Bosmer dont il sera le Roi. Il compte ainsi renforcer ses sujets contre l'influence du royaume de la reine Alinor, qui comprend l'ensemble de l'archipel de Summerset.
\item
1E 143 - Harald, un nordique libre, se proclame Roi de Skyrim, une cité dans la partie Nord de Tamriel, et annexe toute cette partie glacée du continent.
\item
1E 240 - Le royaume de Skyrim annexe Morrowind et HauteRoche.
\item
1E 242/243 - Une jeune esclave Nedes, Alessia, à l'aide d'alliances avec le royaume de Skyrim et certains seigneurs Ayléides rebelles, lance une offensive pour libérer son peuple. Au bout d'un an, après avoir gagné en forces armées, libéré des territoires en Cyrodiil, loin du centre, Alessia fait marcher ses forces, de toutes les directions, vers le centre de la province. Un assaut sur la tour d'Or Blanc est mené, se transformant vite en siège. Les Ayléides cèdent, se rendent, mais sont pour la plupart massacrés. Les lords rebelles décident alors de laisser Cyrodiil aux Nedes, pour se retirer dans le royaume de Valenwood.

La victoire est totale, et Alessia avoue avoir continuellement prié Akatosh. Elle sera sacrée impératrice de Cyrodiil.
\item
1E 246 - La cité de Daggerfall est fondée par les nordiques qui "civilisaient" les terres à l'Ouest de Skyrim. 150 habitants à l'époque. Aujourd'hui plus de 110 000.

\item
1E 266 - Alors que l'impératrice Alessia dirige d'une manière exemplaire son empire, que la culture nedes devient la plus populaire sur le continent, qu'elle a créé il y a quelques années la religion des 8 divinités, une divinité inconnue apparaît au-dessus de la tour d'Or Blanc. Il se présente comme Shezarr, et affirme être le 9ème dieu du culte qu'Alessia a créé. Et pour le prouver, alors qu'elle est mourante, il en fait une sainte. C'est à ce moment qu'est implicitement fait l'accord entre les Aedra et les humains. Les humains vénèrent les 9 dieux (en fait les 9 Aedra les plus puissants), et ceux-ci protègent Nirn des Daedra). En effet, pendant le début du culte des huit divinités, celles-ci ont senti que plus de monde les vénéraient, plus elles étaient puissantes. Cependant, elles se gardent de l'ébruiter.

Belharza, élu par le conseil impérial, devient le second Empereur de Cyrodiil.
\item
1E 355 - Le clan Direnni, une famille Altmer de plus en plus puissante, étend son pouvoir dans le jeune Empire de Cyrodiil par des manœuvres politiques et des complots.
\item
1E 358 - L'armée de Cyrodiil, dirigée par l'Empereur Ami-El, et l'armée de Skyrim font marche sur HauteRoche pour annihiler le clan Direnni avant qu'il prenne trop de pouvoir, car les attentats contre les autorités de l'Empire et de Skyrim s'intensifient depuis 3 ans.
\item
1E 361 - La religion des 8 divinités devient officiellement le culte des neuf, suite à une apparition spectrale d'Alessia dans le palais de l'Empereur, lui rappelant de vénérer Akatosh en plus des autres divinités sous peine de voir l'Empire sombrer. C'est alors la religion officielle et majoritaire de l'Empire.
\item.
1E 369 - Une guerre de succession commence en Skyrim.
\item
1E 401 - Nerevar et Dumac, deux Chimer de la maison Indoril, unissent leurs forces aux Dwemer pour repousser les nordiques de Morrowind.
\item
1E 416/420 - Alors que les nordiques sont effectivement hors de Morrowind, la guerre de succession s'arrête. Désormais, un conseil des vénérables gérera le royaume.
\item
1E 477 - Cette année-là les Direnni refont parler d'eux, mais sont sous-estimés. Ils prennent et clament comme leur une partie de HauteRoche et une portion de Skyrim.
\item
1E 479/482 - Après 3 ans de guerre, et alors que les Direnni ont une force militaire imposante, les forces de Cyrodiil, seules, sans aide extérieure, parviennent à repousser les Direnni. Ils seront alors tous exterminés, pour que leur menace n'existe jamais plus.
\item
1E 700 - Les Dwemer sont chassés par les Chimer (guidés par Dumac et Nerevar, roi de Morrowind), c'est la guerre du premier conseil. Nerevar est trahi par Almalexia, sa femme, ainsi que Sotha Sil et Vivec. Ces derniers utilisent le coeur de Lorkhan et la technologie Dwemer pour obtenir des pouvoirs divins, ce que Nerevar leur avait interdit. À la fin de cette guerre, alors que les combats sont en cours, les Dwemer disparaissent tous, et les Chimer deviennent Dunmer, maudits par Azura, une Daedra, alors que Nerevar est tué. Almalexia, Sotha Sil et Vivec deviennent alors les Tribuns, demi-dieux dirigeants du Tribunal, et accomplissent de grandes choses.
\item
1E 800 - Les Orsimers apparaissent, provenant d'une cité appelée orsinium. Ils sont craints et tués par les humains, alors que les mers les ignorent.
\item
1E 808 - Les rougegardes débarquent dans le territoire qui s'appelle aujourd'hui Hammerfell, en en chassant les nordiques et les humains.
\item
1E 950 - L'assaut sur Orsinium débute, mené par les forces de la cité de Daggerfall.
\item
1E 980 - Orsinium est vaincue. Les civils Orsimers fuient dans les montagnes de HauteRoche.
\item
1E 2200 - Un fléau venant d'un peuple nommé Sloads, une maladie appelé peste trassienne, décime plus de la moitié de la population de Tamriel. Mais ce n'est qu'un prélude.
\item
1E 2703 - En effet, 500 ans plus tard, alors que la maladie et les divers problèmes ont miné la population et les dirigeants, les hommes serpents du royaume d'Akavir attaquent directement Cyrodiil par les côtes (du côté d'Anvil). Cette même année voit pour la première fois tous les peuples de Tamriel s'unir, car les Akavirois ont en tête de conquérir tout le continent. Les Akavirois sont défaits dans la zone contiguë à Cyrodiil, HauteRoche et Skyrim. Après la guerre, les peuples souhaitent rester indépendants, mais reconnaissent la capacité à gouverner et à diriger des Nedes, qui s'appellent eux-mêmes désormais impériaux, car ils savent désormais que leur sang contient la force de diriger et de faire de grandes choses. Les royaumes deviennent des provinces, les routes commerciales s'ouvrent. L'Empire de Cyrodiil, plus de 1000 ans plus tard, deviendra l'Empire de Tamriel.
\item
1E 2813 - le Cyrodiilique devient la langue officielle de l'Empire de Cyrodiil.
\item
1E 2837 - Création de la province des Marais Noirs.
\item
1E 2920 - Une année horrible. Le second assaut des Akavirois est infiniment plus intense et les Akavirois prennent le contrôle de l'Empire, alors que Mehrunes Dagon, un Daedra, détruit la ville de LongSanglot, capitale de Morrowind. Almalexia et Sotha Sil, deux des tribuns, ne peuvent rien faire pour la ville mais, grâce à leurs dons (ils sont devenus demi-dieux), parviennent à bannir Mehrunès de Nirn.
\end{itemize}

\section{2ème ère (-1329 / -434)}

\begin{itemize}
\setlength{\leftmargin}{35pt}
\setlength{\itemsep}{20pt}
\item
2E 1/220 - Cette seconde ère est marquée par le diktat Akavirois. Le conflit entre les hommes-serpents qui ont conquis Cyrodiil et les autres provinces ne dure que peu de temps. Les Akavirois promettant une plus grande indépendance des provinces, les différents rois des provinces n'agissent pas. Les humains sont à nouveau en esclavage.
\item
2E 230 - La première guilde des mages est créée. En effet les Akavirois, ne connaissant pas la magie, forment des guildes pour que les membres de leur race soient formés par des mages humains.
\item
2E 283 - L'Empire d'Akavir demande la dissolution des armées des provinces, assurant garantir la sécurité pour tous. Ceux qui refusent voient leur armée anéantie par les forces Akaviroises.
\item
2E 309 - Création de la province d'Elsweyr, acceptée par les Akavirois, qui ne craignent pas ses félidés bipèdes...
\item
2E 324 - En vacances dans son palais Elsweyrien, l'empereur Akavirois est assassiné par la Morag Tong.
\item
2E 358 - Première citation ou apparition historique de la confrérie noire.
\item
2E 431 - La confrérie noire assassine le nouvel empereur Akavirois et toute sa famille, ainsi que ses héritiers. Devant la recrudescence des assassinats contre ce symbole de la domination Akaviroise, la population commence à préparer une rebellion, très lentement car les autorités redoublent de vigilance.
\item
2E 560 - Un fléau connu comme la peste de Khanaten décime le sud de l'Empire. Bizarrement les argoniens se semblent pas affectés par cette maladie.
\item
2E 572 - Akavir est repoussé de Morrowind. Des rumeurs suggèrent que cela a été rendu possible par les jeunes maisons nobles Redoran et Hlaalu, ainsi que l'aide du demi-dieu Vivec. Les Akavirois décident de quitter Tamriel pour retourner vivre chez eux.
\item
2E 864 - Les rougegardes, cantonnés à l'île de Stros M'Kai, se rebellent contre l'Empire à nouveau humain. Tiber Septim négocie, et Hammerfell revient à nouveau à ces derniers.
\item
2E 882 - Dagoth Ur, un Daedra mineur qui avait été banni par Nerevar, réapparait dans la montagne rouge, avec son armée de créatures. Il parvient à tuer ce qu'il reste des Tribuns hormis Vivec. Une prophétie promet alors le retour de Nerevar, et la défaite de Dagoth Ur. Les pouvoirs Dunmer font construire une muraille magique pour empêcher le fléau propagé par les créatures de Dagoth Ur de s'étendre trop.
\item
2E 896 - Tiber Septim a réunifié les provinces et réunifié l'Empire. La situation est même meilleure qu'elle ne l'était avant l'arrivée des Akavirois.
\end{itemize}

\section{3ère ère (-433 / il y a peu)}

\begin{itemize}
\setlength{\leftmargin}{35pt}
\setlength{\itemsep}{20pt}
\item
3E 38 - Tiber Septim, après pas mal d'années de règne, laisse un Empire glorieux et puissant. Son fils Pelagius Septim lui succède.
\item
3E 41 - Pelagius est assassiné par la confrérie noire. N'ayant pas d'héritier, le trône est confié à sa cousine Kintyra.
\item
3E 48 - Kyntira meurt. Son fils Uriel lui succède.
\item
3E 64 - Uriel meurt et son fil Uriel II hérite du trône. Malheureusement son règne sera rempli d'insurrections, d'épidémies et de problèmes de complots.
\item
3E 65/248 - Une période où les conflits et les guerres succèdent aux problèmes d'insurrections, aux problèmes économiques... Mais dans l'ensemble, une période assez peu troublée, hormis à deux occasions, concernant des batailles pour le trône entre héritiers.
\item
3E 249 - Une armée de Daedra mineurs et de morts-vivants, apparemment commandés par un liche, arrivent d'on ne sait où. Ils conquièrent Valenwood, puis l'Ouest de Cyrodiil, puis Hammerfell. L'empereur d'alors, Cephorus II, envoie toutes les troupes à disposition pour stopper l'invasion de cette horrible armée. Cela stoppe les morts-vivants, mais ces derniers tuent ou transforment nombre de mercenaires.
\item
3E 267 - Ayant vu assez d'atrocités, tous les peuples s'unissent encore une fois, cette fois sous l'égide d'un capitaine de la légion venant de HauteRoche. Après de terribles batailles, le mage de guerre impérial finit par abattre le sorcier liche. Aussitôt l'armée de morts-vivants s'affaiblit, de même que les Daedra mineurs, qui n'ont plus de chefs.
\item
3E 268 - L'année d'après, l'Empire est nettoyé et Uriel V accède au trône. Voulant profiter de l'union de guerre, Uriel V attise la haine de son peuple envers les Akavirois, retournés chez eux. Il mène de grandes batailles sur le territoire pour décimer les clans d'hommes-serpents pouvant y rester...
\item 
3E 288 - Au bout de 20 ans, Uriel V a une aura démesurée, et l'Empire ne s'est jamais aussi bien porté, comme si l'union due à la guerre mettait assez de rivalités de côté pour que de grandes choses soient faites. Alors Uriel V ordonne d'envahir et décimer Akavir. Un nombre impressionnant de navires part, mais les troupes à l'arrivée sont beaucoup moins nombreuses que ce que les Akavirois peuvent soutenir. 
\item
3E 290 - Finalement, malgré le grand nombre d'Akavirois tués grâce aux talents de stratège d'Uriel V, ce dernier finit par mourir lors d'une bataille, et l'on perd les quatre légions qui étaient avec lui. Son fils, trop petit pour diriger, n'accèdera au pouvoir qu'en 307, le mage de guerre impérial faisant la régence entre temps.
\item
3E 313 - Les forces Akaviroises régulièrement décimées, les provinces se réjouissent de leurs relations et de la loi Impériale, rédigée avec les rois de toutes les provinces après six longues années (depuis la prise de pouvoir d'Uriel VI, qui a eu l'âge de gouverner en 3E 307), l'Empire jouit désormais d'une stabilité importante, et les conflits entre royaumes sont un lointain passé.
\item
3E 320 - Mais l'histoire est tumultueuse. Uriel VI meurt d'un accident de cheval, et n'a qu'une fille que le conseil impérial juge tout à fait à même de diriger encore mieux que son père. Morihatha devient impératrice, mais cela soulève des critiques de la part de certaines provinces, qui prennent leur indépendance car Morihatha refuse de déclencher une guerre.
\item
3E 339 - Morihatha est assassinée par la confrérie noire.  Pelagius IV lui succède.
\item
3E 368 - Pelagius IV meurt après un règne où il a réussi à faire réadhérer qusiment toutes les provinces aux valeurs impériales. Uriel VII lui succède et dirige un Empire aussi stable qu'à l'époque d'Uriel V et VI.
\item
3E 389 - Le mage impérial d'Uriel VII, Jagar Tharn, le trahit, l'enfermant dans une dimension qu'il a créée. Jagar Tharn prend le pouvoir.
\item
3E 396 - De nombreuses rebellions menées par des paysans mettent à mal les légions impériales, qui sont souvent en désaccord avec les choix de leur nouveau dirigeant. C'est un usurpateur mais personne n'ose ou ne peut le déloger. Les argoniens et khajiits, réduits en esclavage depuis l'ère d'Uriel V afin d'aider à la reconstruction, se rebellent également. Les légions impériales balaient alors les Marais Noirs et écrasent la révolte.
\item
3E 399 - Jagar Tharn est retrouvé assassiné, Uriel VII est libéré de sa prison. Personne ne sait qui a fait cela, mais Uriel VII reprend vite les choses en main. Pendant ce temps, Orsinium est rebâtie et les Orsimers clament leur indépendance.
\item
3E 414 - Vvarfendell, jusque là occupée uniquement par des Dunmer, est ouverte aux étrangers, à la différence du reste de Morrowind, si l'on ne compte pas Longsanglot, de fait ouverte également à tous. On découvre sur l'île l'ébène et le verre.
\item
3E 417 - On ne sait trop comment, mais la baie d'Iliac, divisée au début de cette année en 44 duchés, baronnies, ou autres royaumes se découpe à la fin de cette même année en 4 : Daggerfall, Wayrest, Sentinel et Orsinium.
\item
3E 427 - L'île de Vvarfendell subit une épidémie de peste venant de la montagne rouge, indiquant l'activité de Dagoth Ur. Cette même année, la prophétie faite des centaines d'années plus tôt se réalise. Un Nerevarine est trouvé, réincarnation de Nerevar. Dagoth Ur est tué, et le ciel s'éclaircit sur l'île.
\item
3E 429 - Almalexia perd ses pouvoirs et, folle de rage, fait assassiner Sotha Sil, qui lui aussi venait de les perdre. D'aucuns prétendent que le Nerevarine aurait reforgé l'épée de Nerevar, ViveFlamme, et aurait tué Almalexia dans les vieilles ruines Dwemer sur lesquelles est bâtie la capitale de Morrowind.
\item
3E 433 - Invasion daédrique sur Tamriel ! Des portes se sont ouvertes un peu partout sur Tamriel. Les Aedra, affaiblis depuis longtemps, ne peuvent rien faire. Il faut un concours de circonstances pour que les portes puissent s'ouvrir, et il faut de la chance et retrouver le dernier héritier du trône pour pouvoir les refermer. Mais Mehrunès Dagon apparaît avant leur fermeture complète dans la cité impériale. Martin Septim, le fil caché d'Uriel VII, se transforme alors en Dragon mythique, avec l'aide de l'Aedra Akatosh, et bannit Mehrunès Dagon de Nirn, en se sacrifiant.

\end{itemize}

\section{4ème ère (de nos jours)}

Les portes vers Oblivion semblent toutes fermées.

Akatosh ayant expliqué que les Daedra ne pouvaient être repoussés par les Aedra faiblissant si l'Empire ne croyait pas fortement en eux, la population les vénére désormais, assurant un maintien de leur pouvoir.

Mais les Daedra ont aussi beaucoup de fidèles et des créatures d'Oblivon, de même que des fanatiques des Daedra, trainent toujours dans l'Empire. Rien n'est terminé, et il n'y a actuellement plus d'Empereur. Le pouvoir en cette première année de la 4ème ère est entre les mains d'Ocato, le mage de guerre impérial qui est actuellement régent, le temps de trouver une solution. Car visiblement, être à ce poste ne l'enchante guère...
