\chapter{Les protagonistes}

  \section{Factions Impériales}
  
    Voici les factions créées ou tolérées partout dans l'Empire
    
    \subsection{Compagnie Impériale}
    
      Présente partout dans l'Empire, c'est l'autorité qui régule le commerce, les taxes entre provinces, qui gère la construction et le maintien des routes, et qui assure l'importation des denrées rares un peu partout dans l'Empire.
      
      Devise : "Partout où il y a des gens, il y a du commerce à faire."
      
    \subsection{Confrérie Noire}
    
      Une faction d'assassins qui se sont séparés il y a de cela longtemps de la Morag Tong, qui tue pour l'argent et sans réfléchir. Ils sont devenus les ennemis jurés de la Morag Tong, qui cherche à les localiser et les tuer tous.
      
      Devise : "Tuer est un art qui mérite salaire, et qui doit être exercé."
  
    \subsection{Culte Impérial}

      Partout dans l'Empire, les prêtres du Culte Impérial (ou Culte des Neuf) sont reconnus et sont souvent écoutés par les dirigeants pour leurs conseils sages et avisés.
      
      Devise : "Les dieux sont avec nous, constamment. Et leurs élus deviennent des héros."

    \subsection{Guildes des mages}
    
      Leur but est d'étudier et de fournir les services liés à la magie à la légion, ou à leurs membres.
      
      Devise : "L'Empire a besoin de mages loyaux et dévoués ? Nous sommes là !"
      
    \subsection{Guildes des guerriers}
    
      Leur but est de gérer l'offre mercenaire dans les différentes provinces. Les mercenaires qui la rejoignent ont gîte, couvert et services, en plus d'avantages sur les tarifs d'armes et d'entrainement. En contrepartie ils fournissent leurs services à la guilde, qui reçoit sans cesse des demandes d'aide pour des tas de raisons, et qui est payée en échange.
      
      Devise : "Le combat peut être un mode de vie. La preuve est faite."
      
    \subsection{Guildes des voleurs}
    
      Leur but est de réguler le commerce et l'activité du vol, afin d'éviter que ce soient toujours les mêmes marchands qui soient dévalisés, ceci afin d'assurer la perennité du commerce. Bien sûr, la guilde garde une partie des bénéfices liés à la revente des biens volés par ses membres, mais en échange fournit gîte, couvert et services.
      
      Devise : "Le vol n'est qu'un autre moyen d'équilibrer les richesses quand il est structuré."
      
    \subsection{Les Lames}
    
      Les yeux et les oreilles du dirigeant de l'Empire. Alors que publiquement ils sont équivalents aux soldats d'élite de l'Empereur, ils sont en fait présents partout en Tamriel et servent d'espions.
      
      Devise : "Notre vie est à l'Empire et à celui ou celle qui le dirige. Nous ferons tout en ce sens, même nous sacrifier."
    
    \subsection{Les Lanternes Jumelles}
    
      Une faction active partout où il y a encore des esclaves. Leur but est de les faire libérer, mais les dirigeants de la faction ne se mettent jamais d'accord sur la méthode, et les membres agissent au final un peu comme ils le veulent.
      
      Devise : "L'esclavagisme est une antiquité. Et maintenant, que fait-on pour que ça change ?"
  
  \section{Factions Dunmers}
  
    \subsection{Camonna Tong}
    
      Syndicat du crime en Morrowind, composé exclusivement de dunmers, les plus xénophobes qui soient, la Camonna Tong est contre toute présence autre que dunmer en Morrowind, et agit en conséquence. La maison noble Hlaalu, bien qu'impérialiste, est actuellement contrôlée par la Camonna Tong, qui peut imposer désormais son veto aux décisions du conseil Hlaalu. La guilde des voleurs est bien entendu l'ennemi naturel de la Cammona Tong, qui tentent souvent de tuer le commerce entre Morrowind et l'Empire.
      
      Ils ont pour habitude, étant concurrents de la Morag Tong, de tuer des membres de cette organisation quand l'occasion se présente.
      
      Devise : "Tous les moyens sont bons pour se débarrasser des étrangers."
    
    \subsection{Morag Tong}
    
      Ce collectif d'assassins est le plus efficace et le plus discret qui soit ! Des membres de la Morag Tong ont assassiné des gens très importants, tout au long de l'histoire de Tamriel. Contrairement à leur ennemi juré, la confrérie noire, la Morag Tong choisit ses contrats. En Morrowind, un meurtrier pris sur le fait est grâcié s'il a un contrat valable et en bonne et due forme de la Morag Tong.
      
      En dehors de Morrowind, la Morag Tong n'est pas officiellement reconnue. Cela ne veut pas dire que fuir Morrowind vous sauvera si un contrat est mis sur votre tête...
      
      Deviser : "Tuer oui. Mais pas n'importe qui, et pas n'importe comment."

  \section{Maisons nobles Dunmers}
  
    Attention ! Il est très difficile de rejoindre une maison noble dunmer. Bien qu'il y ait quelques étrangers membres d'une de ces maisons, parfois même haut placés, ce sont des cas exceptionnels, et les tentatives d'assassinat contre ces exceptions sont assez nombreuses ...

    \subsection{Drès}
  
      Une maison beaucoup moins puissante qu'avant, qui use depuis de moyens pas toujours très légaux pour tenter de récupérer son pouvoir d'antan. Ils ont tendance à essayer de manipuler tous les étrangers qui viennent en Morrowind, quitte à leur mentir sur les raisons des choses qu'ils peuvent leur demander.
      
      Deux points nuisent à sa réputation, et causent la méfiance des étrangers : non seulement la maison Drès abuse d'esclaves dans toutes ses plantations agricoles (c'est la maison qui en exploite le plus, elle fournit même les autres maisons !), mais elle prône toujours les anciennes valeurs dunmers, à savoir le culte des Daedra et le rejet de l'Empire.
      
      Devise : "Un bon esclave travaille la terre. Un mauvais esclave nourrit la terre."
    
    \subsection{Hlaalu}
    
      C'est avec cette maison que l'Empire traite pour le commerce et les relations diplomatiques, les Hlaalus étant élevés pour ça depuis des générations, c'est avec eux que les étrangers se sentent le plus rapidement à l'aise. Ils respectent malgré tout les traditions dunmers, et sont de beaux parleurs et de fins marchands.
      
      Devise : "Si une occasion de faire de l'argent se présente, il faut la prendre. Si elle ne nuit pas à votre réputation."
    
    \subsection{Indoril}
    
      C'est la plus grande maison noble dunmer. Le roi actuel de Morrowind est de cette maison. Ce sont de grands guerriers, bien que les dirigeants sachent parfaitement se mouvoir sur la scène politique. Cette maison est à l'image des dunmers : polyvalente, rusée et puissante.
      
      Ils souhaitent que les dunmers restent dunmers, et rien d'autre. Ce sont eux qui composent la garde de Vivec, une des plus grandes cités de Morrowind.
      
      Devise : "La culture impériale n'est pas la nôtre. Nous sommes dunmers."
    
    \subsection{Redoran}
    
      Une maison noble composée de guerriers. Les Rédorans sont ceux qui recrutent et forment les armées dunmers. C'est également eux qui dirigent les troupes en cas de conflits.
      
      Ils dirigent les cités proches du volcan, dont la plus grande, Ald'ruhn, constituée de squelettes d'une espèce de crabes géants disparus, abrite le siège de leur maison.
      
      Leur principal but, leur raison d'exister, c'est de s'assurer de la perrenité des valeurs guerrières dunmers.
      
      Devise : "Une vie calme et sans combats honorables ne vaut pas la peine d'être vécue."
    
    \subsection{Telvanni}
    
    Cette maison noble est quasi-exclusivement composée de mages dunmers extrêment compétents mais aussi extrêmement méfiants, arrogants et égocentriques. De plus leur mode de vie est singulier : enfermés dans leurs tours champignons près du fort Impérial de Sadrith Mora, en Morrowind, ils n'en sortent quasiment jamais.
    
    Y entrer est non seulement très difficile, mais y rester l'est encore plus, tant les rivalités internes sont importantes !
    
    Le choix d'intégrer un nouveau membre nécessite moult réunions du conseil des 5 plus grands mages de la maison. Pire que les douze travaux !
    
    Enfin, comme si cela ne suffisait pas, les Telvannis sont les plus opposés à l'abolition de l'esclavage après les Drès, car l'esclavage est pour eux un des droits fondamentaux du peuple dunmer !
    
    Devise : "La sagesse donne le pouvoir. Le pouvoir donne le droit."

  \section{Autres Factions}
  
    Au MJ, vous, de piocher dans les factions listées sur le site \url{http://www.uesp.net} et de les intégrer. Cela se faisant dans le cadre, bien souvent, de la création de scénario ou de campagne, pensez à en profiter pour développer vos connaissances sur les régions ou les endroits que vous personnages visiteront. Ce site est véritablement une mine d'informations précieuses.
