\part*{Avant toute chose}

\chapter*{Licence}

Tamriel est une libre adaptation de l'univers de la saga The Elder Scrolls ©, propriété de Bethesda Softworks. Le résultat est un \textbf{univers de jeu de rôle}, sous licence \textbf{Creative Commons BY-NC-SA} dont le détail est à cette adresse : \url{http://creativecommons.org/licenses/by-nc-sa/3.0/deed.fr}.

Cette licence vous autorise à diffuser, modifier ce document, du moment que vous citez son auteur et que vos modifications soient sous cette même licence. C'est une licence virale. La seule interdiction est qu'aucun usage commercial de ce document n'est autorisé.

Les icônes des encadrés font partie du pack d'icônes "ecqlipse 2" de chrfb, dont la galerie DeviantArt se trouve ici : \url{http://chrfb.deviantart.com/gallery/}. Elles sont en Creative Commons By : \url{http://creativecommons.org/licenses/by/3.0/deed.fr}.

Les bordures de la page et les images entourant les titres des chapitres sont des réalisations issues de DeviantArt, dont la galerie se trouve là : \url{http://ascariosa.deviantart.com/gallery/}.

Le fond de page est une création d'Aramisdream issue du pack "Delicate Grunge Paper vol.2" sur Deviantart, dont la page se trouve ici : \url{http://aramisdream.deviantart.com/art/Delicate-Grunge-Paper-vol-2-162580211}

Toutes ces images sont la propriété de leurs auteurs respectifs et sont utilisées ici dans le cadre des autorisations d'utilisation indiquées par les auteurs. Avant toute réutilisation, assurez-vous de bien respecter les conditions d'utilisation de leur travail.

La dernière version pdf de Tamriel est disponible à l'adresse suivante :

\begin{center}
  \url{http://trac.fusina-jdr.org/export/HEAD/branches/tamriel/Tamriel.pdf}
\end{center}

Les sources (au format \LaTeX) sont disponibles à l'adresse suivante :

\begin{center}
  \url{http://trac.fusina-jdr.org/browser/branches/tamriel/}
\end{center}

\clearpage
\chapter*{Guide de Lecture}

Pour faciliter la lecture, certaines zones ont été encadrées. Ces zones illustrent ce qui est écrit dans le reste du document. Un petit logo en haut à droite de ces zones indique leur type.

\exemple{Exemple}{
    Ce type de zone indique une ou plusieurs mises en situation illustrant un point (de règle ou non) défini au dessus.
}

\scenario{Idée de scénario}{
    Ce type de zone indique une idée exploitée en scénario. Ce peut être un court synopsis, une idée de personnage, ou simplement une idée de scène.
}

\note{Note de conception}{
    Ce type de zone indique une note de conception. Les notes de conceptions servent à expliquer pourquoi nous avons fait tel ou tel choix, tant au niveau des règles que de l'univers. Elles permettent au lecteur de mieux sentir l'objectif présent derrière un élement.
}

\option{Règle optionnelle}{
    Ce type de zone indique des règles facultatives, pouvant mieux convenir que les règles "de base" dans certains types de parties par rapport à l'univers choisi par le MJ pour ces parties.
}

\remarque{Remarque}{
    Ce type de zone contient des remarques, des astuces ou des avertissements pour que le futur MJ puisse éviter certains problèmes que nous avons nous mêmes déjà rencontrés.
}
