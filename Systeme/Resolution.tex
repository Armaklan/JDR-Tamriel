\chapter{La Résolution d'une action}

La résolution d'une action intervient quand la réussite des actes d'un personnage est incertaine. Si l'échec ou la réussite sont obligatoires, nul besoin de mécanique ! Quand un doute demeure par contre, l'utilisation du système de jeu s'impose alors. Le système décrit ci-dessous vaut pour tous les actes, qu'ils soient physiques, sociaux ou intellectuel.

\section{En résumé}

La résolution d'une action se déroule comme suit : 

\begin{itemize}
\item Le joueur indique l'action qu'il veut effectuer.
\item Le MJ indique au joueur la caractéristique et la compétence appropriées.
\item Le MJ fixe un facteur de difficulté (action simple) ou effectue le jet pour le PNJ (opposition).
\item Le joueur lance l'intégralité des dés correspondants aux éléments choisis. Il choisit un résultat (en général le plus élevé) et l'annonce au MJ.
\item Le MJ annonce le résultat de l'action (réussite ou échec avec éventuellement des nuances).
\item Le joueur décrit l'action.
\end{itemize}

\section{En détail}

\subsection{Annonce de l'action}

Durant cette phase le joueur doit annoncer ce qu'il désire réaliser. L'important ici n'est pas de décrire avec précision le "comment" mais uniquement l'objectif de l'action. Le système de jeu va permet d'évaluer la réussite de l'objectif, le "comment" n'est qu'affaire de description et vient donc après le jet.

\subsection{Choix des éléments utilisés}

Le maître du jeu doit ensuite fixer les éléments de la fiche de personnage qui vont être utilisés. En général le maître du jeu procède ainsi :

\begin{itemize}
\item Il annonce la caractéristique utilisée en fonction de l'action entreprise. S'il hésite entre deux caractéristiques (cas assez rare), il prend alors la caractéristique la plus avantageuse pour le personnage.
\item Il annonce alors le domaine de compétence à utiliser (discrétion, natation, ...). Le joueur lui propose alors la compétence qu'il juge la plus appropriée. Le MJ est alors libre de l'accepter ou de la refuser. Si le joueur pense n'avoir aucune compétence, alors il fera le jet uniquement avec sa caractéristique ou ses équipements.
\item Si le joueur veut mettre en avant le coté négatif d'un trait, il l'indique alors au maître du jeu.
\end{itemize}

\subsection{Définir la difficulté}

Le MJ doit définir un seuil de difficulté vis à vis de l'action entreprise par le personnage. Ce seuil de difficulté est "indépendant" du personnage qui l'entreprend. Les contraintes dues à l'environnement (difficulté de concentration, temps limité, ...) ne sont pas à prendre en compte dans la difficulté. Elles seront exprimées autrement (voir chapitre "Les Faiblesses").

L'échelle de difficulté (indicative) que nous conseillons est la suivante :

\begin{itemize}
\item Action enfantine : 1
\item Action facile : 3
\item Action malaisée : 5
\item Action difficile : 8
\item Action héroïque : 11
\item Action quasi impossible : 13 ou plus.
\end{itemize}

\subsection{Réussite et échec des actions}

Tout n'est pas binaire dans la vie. Il y a des nuances dans la réussite ou l'échec d'une action entreprise par quelqu'un. Il en est de même dans la réussite des actions tentées par les personnages. 

\begin{itemize}
\item Réussite Totale : L'action qu'entreprend le personnage est réussie, sans aucun soucis.
\item Réussite Partielle : L'action qu'entreprend le personnage est au choix réussie, mais avec des soucis, ou n'est pas réussie mais le personnage se rapproche de son objectif.
\item Réussite de justesse : L'action qu'entreprend le personnage est réussie, mais avec de graves soucis.
\item Échec Partiel : L'action qu'entreprend le personnage est au choix ratée, mais le personnage ne s'éloigne pas de l'objectif, ou alors elle réussit mais le personnage n'en tirera pas d'avantage pour se rapprocher de son objectif.
\item Échec Total : L'action qu'entreprend le personnage est ratée, de plus il s'éloigne de son objectif ou s'attire des ennuis.
\end{itemize}

On considère la réussite ou l'échec en fonction de la différence entre la difficulté et le résultat du jet sur cette échelle :

\begin{itemize}
\item Le résultat est de plus de 3 points supérieur à la difficulté : c'est une réussite totale.
\item Le résultat est entre 1 et 3 points supérieur à la difficulté : c'est une réussite partielle.
\item Le résultat est égal à la difficulté : c'est une réussite de justesse.
\item Le résultat est entre 1 et 3 points inférieur à la difficulté : c'est un échec partiel.
\item Le résultat est de plus de 3 points inférieur à la difficulté : c'est un échec total.
\end{itemize}

Ces seuils sont indicatifs. Avec l'expérience, vous apprendrez à faire peser chaque point, chaque marge différente, dans la balance pour évaluer le résultat d'une action.

\subsection{Description du résultat}

Une fois que le MJ a déterminé la réussite ou l'échec de l'action, c'est au joueur de décrire l'action en détails et son résultat. Le joueur est libre de la décrire et de l'interpréter comme il l'entend. Le MJ peut, toutefois, décider à tout moment de mettre son veto sur un élément particulier, ou de compléter la description.

La description doit bien sûr être en accord avec les résultats obtenus. Si ce n'est pas le cas, il est évident que le MJ doit la reprendre intégralement.

\exemple{Escalade nocturne}{

Lydia poursuit une bande de brigands qui possèdent des informations vitales pour son ordre. Les brigands l’ont conduite vers ce qui semblait être un cul de sac. Par contre, grâce à leur agilité, ils ont réussis à escalader un mur et à pénétrer dans le jardin d’un des hôtels. Lydia est bien décidée à les suivre !

En bref, cette action est incontestablement physique (où Lydia a d8). Par contre, elle n’a aucune compétence concernant l’escalade... 
Le mur fait 2m de haut, avec pas mal d’irrégularité facilitant l’escalade. Le MJ considère qu’elle est plutôt facile et attribue une difficulté de 3. Le joueur de Lydia lance alors un d8 et obtient 5. On compare ce résultat à la difficulté. Il y a une différence de 2, c’est donc une réussite partielle.

Le joueur doit alors décrire le résultat. D’une manière générale, l’action est réussie (il passe au-dessus du mur) mais il y a une complication. Le joueur choisit de lui-même (avec l’accord du MJ) ce qui viendra qualifier sa réussite de partielle.

- " A la vue des différentes prises offertes par les pierres, Lydia commence à escalader à son tour le mur. Elle compte principalement sur sa force pour le hisser en haut. L’escalade n’est pas aussi coulante que celle des brigands mais elle s’en tire plutôt bien. Arriver en haut par contre, c’est une autre histoire. Les pierres du haut sont mal attachés et, dans ma descente, elle en entraîne plusieurs avec elle. Sûr qu'elle a réveillé tout le voisinage avec ce raffut ! " 

}

\section{Porter assistance}

Parfois, plusieurs personnages travaillent ensemble à la résolution d'un objectif commun. Dans ce cas, un des personnages va réellement exécuter l'action finale, on le nomme alors le "Meneur". Les autres personnages viennent alors juste apporter leur aide à ce meneur, avec leurs compétences propres.

Quand un tel cas de figure se présente, seul le meneur va effectuer le jet permettant de savoir si l'action est réussie ou au contraire échouée. L'aide des autres joueurs va permettre d'apporter des bonus augmentant les chances de succès. Ce bonus se présente sous la forme d'un (ou plusieurs) dé(s) supplémentaire(s) correspondant(s) au(x) dé(s) de compétence de la (ou des) personne(s) assistant le meneur.

Pour évaluer la réussite de l'action, le meneur va lancer :

\begin{itemize}
\item son dé de caractéristique
\item son dé de compétence
\item le dé de compétence des personnages qui l'assistent
\end{itemize}

Tous les personnages peuvent décider d'utiliser des traits ou du stress sur cette action. Le reste de la résolution se passe de façon standard.

\section{Les traits}

Les traits d'un personnage peuvent intervenir en jeu pour toutes actions en rapport. Par exemple, un trait "Soldat de l'Empire" pourra intervenir sur les actions militaires, mais aussi sur les actions en rapport avec le statut du personnage.

Il existe deux types d'utilisation :

\begin{itemize}
\item Comme avantage : il est utilisé pour aider le personnage à accomplir son action.
\item Comme désavantage : il est utilisé pour mettre en difficulté le personnage, soit en le pénalisant dans une action, soit en amenant le personnage à se placer dans une situation difficile.
\end{itemize}

\subsection{Comme avantage}

Un trait peut être utilisé lorsqu'il correspond à l'action entreprise. L'utilisation ou non d'un trait peut être discutée entre les joueurs et le MJ.

Les traits peuvent être ceux du personnage qui peuvent l'aider dans l'action qu'il tente, mais également des éléments Descriptifs de l'environnement (des débris pour se cacher, un tuyau par terre comme arme improvisée, un lustre pour échapper au soldat ennemi, etc...) !

\exemple{Exemples d'utilisation de trait}{

\begin{itemize}
\item Le trait Forte-Tête peut être utilisé pour résister à une tentative de Persuasion. Par contre, il ne peut pas être utilisé pour gagner une course de vitesse (sauf si le joueur trouve vraiment une explication béton, dans ce cas, acceptez-la car il faut favoriser l'inventivité des joueurs).
\item Le trait Agilité Féline peut être utilisé pour tenter une acrobatie périlleuse, pas pour effectuer un travail mental !
\end{itemize}

}

Un trait utilisé permet de :

\begin{itemize}
\item Relancer tout ou partie d'un jet de dés
\item Obtenir un bonus de +2 sur le résultat total
\end{itemize}

Une fois utilisé, le joueur doit effectuer une coche à coté du trait. Cette marque permet d'indiquer que le trait a déjà été utilisé. Il ne pourra plus l'être par la suite sauf si le joueur obtient une décoche (cf Regain de trait) ou si il dépense un point de Stress.

Dans certains univers à l'ambiance plus héroïque, le maître du jeu pourra autoriser le joueur à cocher 2 fois chaque traits avant de le rendre inutilisable.

\subsection{Comme désavantage}

Un trait peut également être utilisé pour pénaliser le personnage : 

\begin{itemize}
\item soit en amenant le personnage à se mettre en difficulté : foncer sans réfléchir sur le grand méchant à cause de sa haine pour lui, plonger dans un piège à cause de son insouciance, ou tout simplement attirer les suspicions à cause de ses origines.
\item soit en pénalisant le personnage lors d'une action : sa peur du vide qui lui rend difficile la traversé d'un pont qui passe au dessus de la vallée, sa fascination pour les femmes qui l'empêche de repérer le mensonge dans la bouche d'une jolie dame, ... Dans ce cas, le joueur réduit tous les dés lancer d'un échelon (D6 devient D4, D10 devient D8, et D4 disparaît).
\end{itemize}

Si la mise en avant du trait comme désavantage à un intérêt scénaristique, le MJ vous gratifiera d'une décoche (possibilité de décocher un trait déjà coché). 

\subsection{Décocher un trait}

Il existe trois façon d'obtenir des décoches permettant de réutiliser à nouveau un trait qui avait été coché :

\begin{itemize}
\item Utiliser un désavantage (comme expliquer au chapitre précédent).
\item Effectuer une interprétation exceptionnelle de votre personnage. Quand vous jouer votre rôle avec soin et que vous participez à établir une bonne ambiance autour de la table, le maître du jeu est incité à vous récompenser par une décoche immédiatement. Ce gain devra toutefois rester exceptionnel et gratifier un réel effort de la part du joueur.
\item Lors des scènes de repos (longue nuit de repos, repas tranquille dans l'auberge du coin, ...), le maître du jeu offrira une décoche pour les joueurs ayant contribué à faire avancer le scénario dans les scènes précédentes. Si vous avez bien interpréter votre personnage, que vous avez effectuez quelques actions utiles ou soumis de bonnes idées, vous serez donc récompensé.
\end{itemize}

\subsection{Le Stress}

Les points de Stress sont une réserve de points directement à disposition des joueurs. Tous les joueurs peuvent, quand ils le désirent, utiliser des points de Stress. Une fois les points de Stress utilisés, ils sont mis à disposition du MJ. Le MJ pourra alors les utiliser pour ses PNJ et les remettra ensuite dans la réserve générale.

En général, le nombre de points de stress disponible est égale à deux fois le nombre de joueurs présents. Le maître du jeu pourra bien sur faire varier cette réserve selon l'ambiance qu'il désire obtenir.

Les points de Stress ont différentes utilisations :

\begin{itemize}
\item Utiliser un trait déjà coché
\item Faire abstraction des ses blessures
\item Influencer la narration
\item Utiliser une compétence complémentaire
\end{itemize}

\subsubsection{Utiliser un trait déjà coché}

Un joueur peut avoir envie d'utiliser un trait qu'il a malheureusement déjà coché. Les points de Stress lui permettent de se dépasser en utilisant le trait malgré la coche.

L'utilisation du trait se passe comme une utilisation standard.

Attention toutefois, un trait ne peut être utiliser qu'une seule et unique fois sur une même action : impossible de dépenser plusieurs point de stress sur le même trait, ou de dépenser un point pour utiliser un trait qui vient tout juste d'être coché.

\subsubsection{Influencer la narration}


Cette option doit être validée par le maître du jeu avant utilisation. Nous incitons toutefois les maîtres du jeu à l'autoriser car elle donne aux joueurs un peu plus de pouvoir et permettent parfois des rebondissements originaux. Bien sûr, à chaque utilisation, le MJ est libre de donner son veto.

Influencer la narration permet aux joueurs de dépenser des points de stress pour compléter la description du MJ. Il ne s'agit pas de venir contredire ce que le MJ a déjà dit, mais bien de préciser des choses qu'il n'a pas encore dites. Selon les cas, le MJ peut demander au joueur de dépenser 1 (influence légère), 2 (influence moyenne) ou 3 (influence vraiment importante) points de stress.

Un nouvel élément Descriptif de scène peut être créé suite à la description du joueur. Dans ce cas, la première utilisation que le joueur fera de ce nouveau Descriptif de scène est considérée comme gratuite.

\exemple{Exemple d'influence sur la narration}{

\begin{itemize}
\item Le MJ décrit que le joueur, passant de ruelle en ruelle, tombe soudain sur un cul de sac. Le joueur peut alors dépenser 1 point pour indiquer qu'une échelle est posée le long du mur, et que, grâce à elle, il peut se sortir de cette impasse !
\item Coincé dans une cave, le joueur explique au MJ qu'il compte tenter d'ouvrir la porte avec une pièce de métal trouvée dans un recoin sombre de la pièce.
\end{itemize}

}

\subsubsection{Faire abstraction des ses blessures}

Cette option permet au joueur de ne pas être pénalisé par ses blessures pour une action donnée. En dépensant un point de Stress, le joueur peut effectuer son jet sans pénalité. L'annulation ne vaut que pour l'action en cours : la faiblesse pénalisera à nouveau le joueur dès la prochaine action.

\subsection{Compétence complémentaire}

Parfois, un personnage possède une compétence qui peut l'aider à accomplir une action, sans toutefois être la compétence principale. On appelle cette compétence une compétence complémentaire. 

Si le joueur veut faire intervenir cette compétence complémentaire dans un jet de dé, il peut le faire en dépensant un point de stress. Il ajoute alors le dé de la compétence en plus de ces dés classiques, comme il l'aurait fait dans le cas de "Porter une assistance".