\chapter{Les descriptions}

\section{Résoudre avant de décrire}

Les vieux routards ont un réflexe basique : ils décrivent leurs actions, ils les résolvent, puis le MJ décrit à son tour. Chaque action passe donc par deux phases de description, une de l'attendu du joueur (ce qu'il souhaite faire) et ensuite celle de l'action réellement faite par le personnage (le résultat). 

La plupart des jeux fixent la difficulté également en fonction des détails que donne le joueur de son action. Certains joueurs se retrouvent donc à éviter de trop décrire et de trop en rajouter de peur d'obtenir des malus qui pourraient les amener à échouer alors qu'ils essayent juste de réaliser une action spectaculaire, digne des films d'action.

Fusina est conçu pour casser ce schéma de la manière suivante :


- Le joueur annonce son objectif, afin que le MJ puisse fixer une difficulté et indiquer le jet au joueur
- On résoud l'action, le MJ annonce de manière succinte le résultat de l'action au joueur.
- Le joueur décrit l'action.


La seule description de l'action se fait une fois le résultat connu. La description peut donc se concentrer sur ce qui est vraiment le résultat réel de l'action. On gagne en fluidité, et en précision. De plus, les joueurs peuvent savoir à quel point en rajouter en fonction de ce que le MJ leur dit (car le MJ indique dans sa description concise si l'action est réussie, ratée, de justesse ou pas).

Toutefois, ce point me mène directement au second conseil.

\section{Laisser les joueurs décrire}

Point très important, le MJ doit casser son monopole de la description du résultat des actions. Le MJ devrait se contenter d'annoncer froidement et de manière concise le résultat "technique" d'une action pour laisser aux joueurs libre court à leur imagination dans la description de ce qu'il se passe. Le MJ conserve alors un rôle d'arbitre. Il peut décider de contredire une description s'il juge qu'elle n'est pas en phase avec l'univers ou le résultat technique, la compléter pour donner de la consistance. Mais le narrateur de l'action reste bel et bien le joueur.

Quels sont les avantages de cette méthode ?


- La description colle davantage au personnage. Un réel style peut être attaché à chaque action du même personnage.
- Les joueurs ont des idées que le MJ n'aurait pas eu.
- Les joueurs peuvent décrire des actions dignes des films (le héros qui se pend au lustre pour tomber sur son adversaire par exemple) sans limitation. Le résultat est déjà connu, la description n'est que panache.


\section{Penser Cinématique}

Voilà, finalement, le conseil le plus important que l'on peut vous donner, et qui correspond vraiment à l'optique de Fusina : Penser cinématique.

Concrètement, qu'entendons-nous par cela ? Nous pensons que la partie de jeu de rôle doit être vécue par les joueurs comme le serait un film de l'ambiance choisie. Attention, cela ne veut pas dire que les joueurs sont spectateurs, mais tout simplement que les joueurs interprétent les personnages principaux d'un film, ils sont le centre de l'histoire. 

Dans vos descriptions (mais aussi dans celles que les joueurs vont être amenés à faire), essayez de toujours penser à comment cela serait représenté dans un film. Imprégnez-vous avant les parties de films appropriés à votre univers, imprégnez-vous des angles de caméra, des scènes spécifiques, des odeurs qui ressortent de l'écran. Réunissez tous ces élements et exploitez-les dans vos séquences.

Et comment faire pour que les joueurs utilisent aussi cette vision dans leurs descriptions ? Guidez-les ! Proposez-leur des élements au fur et à mesure de leurs descriptions, complétez-les dans ce sens. Petit à petit, ce type de description se fera naturellement pour eux. 

Ah, et un dernier conseil... Ne cassez surtout pas le rythme ! Un doute sur le système ? Faites simplement comme vous le sentez, Fusina est adapté à une gestion au feeling, alors lâchez-vous ! Vous ne savez pas comment interpréter la gêne occassionée par ce champ de fumée, demandez juste au joueur de noter sur leur feuille une faiblesse "fumée" qui disparaîtra dès qu'ils en seront sortis. Vous ne savez pas comment gérer l'aspect fatigue ? Pareil, faites noter une faiblesse "Fatigue", et hop, c'est reglé ! Faites comme vous le sentez, sans jamais interrompre l'histoire et son déroulement, sans jamais laisser votre ambiance ou le rythme retomber.

\chapter{La Création de Personnage}

\section{Posez des questions}

La création des personnages est un moment crucial pour leur "vie" à venir en jeu. Le background des personnages n'est pas uniquement fait pour servir de décor, d'historique. Il s'agit d'un ensemble de faits et d'acteurs qu'il faudra intégrer, coûte que coûte, au scénario. Alors, ne laissez pas les joueurs le construire seuls dans leurs coin.

À la fin de la création des personnages, demandez successivement à chaque joueur de décrire son personnage. Et, dès que vous avez la moindre opportunité, harcelez-le de question ! Votre but ? Faire ressortir de son background des personnages, des amis, des rivaux, des ennemis, bref, toute personne qui pourrait apparaitre dans les scénarios d'une manière ou d'une autre. Certains personnages tenteront de nuir au pj, d'autre tenteront de l'aider, d'autre agiront comme des faiblesses capables d'attirer les pires ennuis au personnage, et enfin, d'autres gêneront le pj par leur simple présence, sans même réellement intéragir dans le scénario. 

À chaque apparition, le personnage du joueur prendra de la profondeur.

\section{Faites-le en groupe !}

Ne créez jamais les personnages de façon isolée, faites-le en groupe. Laissez les autres joueurs intervenir dans la séance de questions. Ils poseront peut-être des questions auxquelles vous n'avez pas pensé. Parfois ils pourront lier leurs personnages par un élement commun. Ils auront des bonnes idées qu'ils pourront soumettre. Tout le monde y gagnera, et vous aurez un vrai groupe de personnages joueurs.

Et comment faire pour les choses que les joueurs veulent garder secret ? Aborder ces points, et uniquement ceux-ci, en solo avec le joueur. Mais évitez d'en avoir trop tout de même. Les personnages sombres et mystérieux ayant tous nombre de sombres secrets, c'est un peu surfait !

\chapter{Amusez vous !}

Le livre touche à sa fin. Le dernier et ultime conseil que nous vous donnons est de vous amuser ! Ne pas se prendre la tête et raconter de belles histoires, faire vivre de superbes aventures dans le monde de la saga "The Elder Scrolls", voilà tout ce que nous vous souhaitons.

     