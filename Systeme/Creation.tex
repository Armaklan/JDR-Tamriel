\chapter{Création du personnage}

Elle est identique à la création standard de Fusina, à quelques détails près.

\section{Niveau d'Héroïsme et niveau de Tension}

Dans Tamriel, le niveau d'Héroïsme est fixé à 2. Cela autorise donc 6 traits, et il y a donc 2 fois le nombre des joueurs Points de Panache en début de partie. Chaque joueur commence une séance avec 1 point de Panache.

Ensuite, le niveau de Tension est également fixé à 2. Cela fait qu'en début de partie, il y aura 2 fois le nombre des joueurs Points de Stress.

Sur une action, un joueur peut donc (à condition qu'il en ait/reste) utiliser 2 points de Panache et 2 points de Stress des façons habituelles dans Fusina ("Utiliser un trait", "Influencer la narration", "Ignorer une faiblesse", etc...).

\option{Et si le premier panache était la description du personnage ?}{
    Testée et aprouvée, cette méthode fait que si vous commencez la traditionnelle description des personnages par le joueur expérimenté, et finissez sa longue description par la distribution d'un point de Panache, cela va fortement motiver les autres joueurs à bien décrire leur personnage également.
    
    Essayez, vous verrez !}

\section{Choix du peuple}

Le joueur a le choix parmi plusieurs peuples. Chacun de ces peuples a ses particularités, mais la plus marquante (indiquée dans la liste suivante à côté du nom de chaque peuple) est un trait à indiquer sur la fiche du personnage. Le défaut le plus courant de chacun des peuples est aussi indiqué, et doit être mis dans "Faiblesse".

Ce choix préremplit donc un des 6 traits auxquels le joueur a droit à la création pour son personnage.

\begin{itemize}
\item
Altmer : Trait " Puits de science", Faiblesse "Extrêmement arrogant".
\item
Argonien : Trait "Insensible aux toxines", Faiblesse "Recherche la liberté pour son peuple"
\item
Bosmer : Trait "A du apprendre à survivre", Faiblesse "Incapable de rester diplomate face à un altmer"
\item
Breton : Trait "Versé dans les arts magiques", Faiblesse "Toujours dans ses bouquins"
\item
Dunmer : Trait "Sait reconnaître les menteurs", Faiblesse "Souvent très mal vu en société"
\item
Impérial : Trait "Leader-né", Faiblesse "Toujours plus de pouvoir !"
\item
Khajiit : Trait "Agilité féline", Faiblesse "Dépendant au skooma"
\item
Orsimer : Trait "Force inouïe", Faiblesse "Tête brûlée"
\item
Nordique : Trait "Ne recule jamais", Faiblesse "Ne recule jamais... devant un combat"
\item
Rougegarde : Trait "Fin stratège", Faiblesse "Sens de l'honneur"
\end{itemize}

Passé ce trait et cette faiblesse, chaque personnage a donc 6 traits à inscrire sur sa feuille.

De plus, le joueur doit trouver une autre faiblesse, spécifique à son personnage, qui donnera droit à un trait supplémentaire.

\section{Caractéristiques}

Aucun changement par rapport à la base de Fusina.

\section{Compétences}

Pour les compétences, 25 points sont offerts sur l'échelle de prix de base dans Fusina.

Les domaines de magie ne sont pas achetés avec ces points, mais les compétences d'Alchimie et d'Enchantement oui.

\section{Domaines Magiques}

Ils se notent dans les compétences, et s'achètent au même tarif que les compétences, mais avec un pool de points dépendant du score d'Intellect selon ce tableau :
\\
\\
\begin{tabular}{|c|c|}
    \hline
    Score d'Intellect & Nombre de points à dépenser \\ \hline \hline
    d4 & 0  \\ \hline
    d6 & 2 \\ \hline
    d8 & 6  \\ \hline
    d10 & 12 \\ \hline
    d12 & 20 \\ \hline
\end{tabular}

Cependant, on ne pourra atteindre dans un domaine magique le niveau d'Intellect. Ce qui signifie qu'un personnage à d4 en Intellect ne pratiquera probablement jamais la magie.

\exemple{Prenons un exemple}
{Comme Jiub a d10 en Intellect, il a 12 points à dépenser pour l'achat de compétences dans les domaines magiques, et il ne pourra atteindre dans un domaine la valeur de d10, mais il pourra avoir un domaine à d8, car cela reste sous sa valeur d'Intellect.}

\section{Faiblesses}

Aucun changement par rapport à la base de Fusina.

\section{Équipement}

Là encore, c'est identique à la base de Fusina. Dosez au feeling.
