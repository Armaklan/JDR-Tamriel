\chapter{La magie}

\section{Introduction}

La magie dans Tamriel est liée à l'énergie présente dans chaque être, la magicka.

Comme dit plus tôt, il existe Sept domaines de Magie :

\begin{itemize}
\item
L'altération : les effets de ce type de magie modifient réellement les choses, voir les déforment. Les modifications sont permanentes pour les choses non vivantes, et temporaires pour les choses vivantes.
\item
La conjuration : ce domaine regroupe tout ce qui concerne l'invocation et le bannissement d'êtres provenant ou non de Tamriel.
\item
La destruction : ici on affaiblit, on blesse, on détruit. C'est le domaine de magie le plus offensif, et cela se comprend !
\item
L'illusion : comme son nom l'indique, ce domaine regroupe tout ce qui est lié à la supercherie, à la tromperie, au faussage des sens.
\item
La nécromancie : ce domaine regroupe tout ce qui concerne la modification de l'état d'un être autrefois vivant. C'est dans ce domaine magique que l'on retrouve également les malédictions et les maladies non naturelles. Ce domaine est très mal vu par toutes les autorités.
\item
Le mysticisme : ce domaine regroupe tous les effets transcendant ou supprimant les limites de la réalité. Voler, se téléporter, voilà qui est pratique bien que dangereux !
\item
La restauration : Réparer, restaurer, protéger, tout cela est du domaine de la restauration.
\end{itemize}

Lié à ces domaines magiques, trois utilisations possibles :

\begin{itemize}
\item
Sous forme brute : C'est l'utilisation de la magie la plus courante. En Destruction, par exemple, ce serait le fait d'envoyer une boule de feu ou des éclairs !
\item
L'alchimie : C'est le fait d'inclure un effet d'un des domaines de magie sous une forme "transportable", afin de permettre à des non mages d'user de magie en buvant, mangeant ou lançant une potion ou un plat cuisiné !
\item
L'enchantement : Contrairement à l'alchimie, on n'inclue pas un effet magique dans un liquide ou de la nourriture, mais directement sur un objet !
\end{itemize}

Il faut savoir que l'on peut utiliser tous les domaines de magie sous chacune de ces trois formes.

\section{Déterminer le sort et sa difficulté}

Tout d'abord, le joueur décrit brièvement son sort.

Cela permet de déterminer le domaine de magie, et différents paramètres, ce qui nous aidera alors à évaluer la difficulté en fonction des effets souhaités :
\\
\begin{tabular}{|c|c|c|c|c|}
    \hline
    Distance     & Durée         & Aire d'effet & Vitesse          & Guérison      \\ \hline
    Toucher :  0 & Aucune   :  0 & Aucune  :  0 & Escargot    :  0 & Nulle    :  0 \\
    1m      :  1 & Scène    :  1 & 1m      :  1 & Humain      :  1 & Légère   :  1 \\
    10m     :  3 & Heure    :  3 & 3m      :  3 & Cheval      :  3 & Grave    :  3 \\
    100m    :  6 & 6 heures :  6 & 10m     :  6 & Flèche      :  6 & Mortelle :  6 \\
    1km     : 10 & Journée  : 10 & 25m     : 10 & Plus rapide : 10 & - \\ \hline \hline
    Concentration               & \multicolumn{2}{|c|}{Poids/Solidité         } & \multicolumn{2}{c|}{Volonté (Cible)                      } \\ \hline
    Les yeux fermés        :  0 & \multicolumn{2}{|c|}{Feuille de papier  :  0} & \multicolumn{2}{c|}{Non vivant                     :  0 (6)} \\
    Ça en demande un peu   :  1 & \multicolumn{2}{|c|}{Humain             :  1} & \multicolumn{2}{c|}{Animaux/Créatures non magiques :  1} \\
    Ça en demande pas mal  :  3 & \multicolumn{2}{|c|}{Porte en bois      :  3} & \multicolumn{2}{c|}{Humains / Morts vivants        :  3} \\
    Ça en demande beaucoup :  6 & \multicolumn{2}{|c|}{Mur de pierre      :  6} & \multicolumn{2}{c|}{Créatures magiques / Spectres  :  6} \\
    C'est impossible       : 10 & \multicolumn{2}{|c|}{Bâtiment en pierre : 10} & \multicolumn{2}{c|}{Aedras, Daedras et autres      : 10} \\ \hline
\end{tabular}

La difficulté du sort est la plus haute des difficultés des paramètres appliqués au sort.

La difficulté est multipliée par le nombre de domaines différents qui composent le sort.

\section{Fatigue}

Chaque utilisation de la magie peut fatiguer le lanceur de sort.

Pour savoir si c'est le cas, on regarde la marge du jet pour le sort effectué. En fonction de cette marge, le lanceur de sort subira plus ou moins le contrecoup du sort, appelé Drain.

Cela est représenté par la dépense de points de Stress comme suit :

\begin{itemize}
\item
Marge de -3 et moins : Le mage a complètement perdu le contrôle de son sort, et en plus de le rater, il y a perdu beaucoup d'énergie ! Il doit dépenser 2 points de Stress pour ne pas subir un trait "Épuisé" qui ne disparaîtra qu'après une période de repos d'au moins 5 heures, et qui comptera pour n'importe quelle action physique.
\item
Entre -3 et 3 de marge : Le mage a limité la casse, mais il reste qu'il a eu les yeux plus gros que le ventre, sort réussi ou pas. La fatigue le gagne, il doit dépenser 1 point de Stress pour ne pas subir un trait "Légère Fatigue" qui disparaîtra à la fin de la scène, et qui comptera pour n'importe quelle action physique.
\item
Plus de 3 de marge : Le mage a très bien géré la magie, et a su tirer partie de ce qui était présent dans son environnement. En conséquence, bien que cela l'ai très légèrement fatigué, il ne subit aucune faiblesse et ne dépense aucun point de Stress.
\end{itemize}

\section{Lancer un sort sous forme brute}

Cela consiste à utiliser un domaine de magie sous sa forme primaire. Une fois la difficulté déterminée comme vu précédemment, on fait le jet, et on considère le résultat.

\section{Faire de l'alchimie ou de l'enchantement}

Tout d'abord, le joueur doit préparer l'objet qui stockera le sort, et sera catalyseur (potion, repas, bâton, dague, vêtement, bijou, ...).

Pour cela, on fait un jet d'Âme avec la compétence Alchimie ou Enchantement selon le cas. Le résultat donne le nombre d'utilisations possibles.

On fait ensuite la création et la résolution du sort comme s'il était effectué sous forme brute.

L'alchimie (ou l'enchantement) n'est réussi(e) que si les deux jets sont réussis. On note alors l'effet exact de l'objet et son nombre d'utilisations possibles pour s'en souvenir. Le joueur et le MJ se mettent d'accord sur la façon d'utiliser la création.

\remarque{Attention aux illogismes !}{
    En alchimie, par exemple, si un sort s'active au bris de la fiole qui le contient, toutes les utilisations seront utilisées, car il est évidemment impossible de rebriser la fiole.
    
    De même, un bâton enchanté d'un sort de restauration pourra soigner son porteur, pas soigner en donnant des coups aux alliés !
}
