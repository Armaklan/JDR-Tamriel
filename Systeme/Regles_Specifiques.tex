\chapter{Introduction}

Tout d'abord, sachez qu'il n'y aura pas des modifications à outrance de la mécanique de Fusina (à vrai dire aucune), juste des précisions.

\section{Armures et Boucliers}

Dans Tamriel, certains combattants, selon leur métier et leur statut (qui se mesure avec la caractéristique Influence, rappelons-le) peuvent avoir des boucliers et/ou des armures.

Les armures/protections fonctionnent de manière identique au système de base de Fusina, avec une valeur fixe indiquant des niveaux de blessures diminuables sur une scène. Par exemple, une armure de protection 2, peut au choix transformer deux blessures en deux blessures d'un niveau moins grave (par exemple deux mortelles en deux graves), ou alors baisser de 2 niveaux une blessure (une mortelle devient une blessure légère, par exemple).

Le bouclier, lui, fonctionne comme l'armure (il peut absorber des blessures) avec un niveau de 1 à 3, sauf que l'on peut s'en servir plusieurs fois par scène, à la seule et unique condition d'indiquer défendre/tenir une position !

\section{Combat à distance}

A la différence du combat de mêlée ou de corps à corps, le combat à distance est vite mortel.

En effet, si l'archer ne cherche pas à être discret, la cible peut faire un jet de perception de Facile à Difficile selon la distance et les conditions de perception.

Si la cible réussit son jet, elle aperçoit l'archer. À partir de là, les flèches et carreaux étant moins rapides que des balles de pistolet moderne, la cible pourra tenter de les esquiver ou de les parer comme une attaque en mêlée.

Les armures peuvent être utilisées pour diminuer les dégâts d'une attaque à distance dans les mêmes conditions qu'en Mêlée.

Cela ne conviendra pas aux fanatiques du réalisme, mais les combats n'en sont que moins déséquilibrés et plus fun !
