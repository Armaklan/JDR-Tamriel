\chapter{Conflits}

\section{Types et déroulement des conflits}

Dans beaucoup de jeux de rôle, quand les Personnages Joueurs (PJ) rencontrent un Personnage Non Joueur (PNJ) agressif, généralement, cela se solde par un conflit. Mais ce conflit n'est pas forcément physique. En effet, tout dépend de ce que les joueurs souhaitent faire. S'ils interrogent le PNJ, cela pourra se régler en un conflit d'Intellect, pour représenter le duel de volonté entre l'interrogateur qui veut l'information, et l'interrogé qui ne veut la donner.

Qu'ils soient physiques, mentaux, sociaux, d'influence, nous gérons dans Fusina tous les conflits de la même manière, mais avec deux degrés de détails : les conflits courts et les conflits importants.

\subsection{Les conflits courts (contre les figurants)}

Ces conflits sont les plus courants : il s'agit de toutes les situations où un personnage affronte (physiquement, socialement, mentalement, etc...) quelque chose (un ennemi, une tâche complexe,...). On ne se préoccupe que de l'action du personnage. En termes de jeu, le joueur indique au MJ l'action qu'il souhaite accomplir.

En fonction de l'action souhaitée, le MJ détermine la difficulté, et fait faire un seul jet au joueur pour déterminer l'issue de la scène. En effet, les conflits courts sont là pour l'action, mais n'ont pas de réelle importance dans l'histoire, la scène est donc résolue en une fois.

\subsection{Les conflits importants}

A l'inverse, le(s) grand(s) méchant(s) compte(nt) beaucoup dans l'histoire, ils ne peuvent pas apparaître uniquement 30 secondes ! Il y a également certaines scènes porteuses d'une forte intensité dramatique qui méritent d'être traitées plus longuement.

Dans ce cas, les actions vont s’enchaîner en faisant progresser les différents objectifs poursuivis par les intervenants. Un but est rarement atteint en une seule action. Il faut en général plusieurs réussites pour réaliser un objectif. 

\section{La résolution d'un conflit long}

\subsection{Définition des termes de bases}

\subsection{Déroulement}

Le conflit est découpé en round. La durée d’un round dépend du type de conflit :

\begin{itemize}
\item Un round de combat va durer une dizaine de seconde
\item Un round d’une bataille va durer plusieurs minutes
\item Un round social pourra durer de longue minute
\item Un round dans un combat d’influence va durer plusieurs heures, voir même plusieurs jours !
\end{itemize}

Lors d’un round, chaque protagoniste va avoir l’occasion d’effectuer une action. Les actions se résolvent selon la règle standard de Fusina. Quand une action va à l’encontre d’un personnage, celui-ci peut se défendre. L’action est alors résolue selon les règles d’opposition simple.

Lors des oppositions, certains personnages vont subir des handicaps, aussi appelés blessures dans le cas de combat. Ces handicaps vont pénaliser le personnage au cours de l'action, mais aussi pour le reste de l'aventure (cf chapitre Handicap et Blessure). 

\subsubsection{Combat à distance}

A la différence du combat de mêlée ou de corps à corps, le combat à distance est vite mortel.

En effet, si l'archer ne cherche pas à être discret, la cible peut faire un jet de perception de Facile à Difficile selon la distance et les conditions de perception.

Si la cible réussit son jet, elle aperçoit l'archer. À partir de là, les flèches et carreaux étant moins rapides que des balles de pistolet moderne, la cible pourra tenter de les esquiver ou de les parer comme une attaque en mêlée.

Les armures peuvent être utilisées pour diminuer les dégâts d'une attaque à distance dans les mêmes conditions qu'en Mêlée.

Cela ne conviendra pas aux fanatiques du réalisme, mais les combats n'en sont que moins déséquilibrés et plus fun !

\subsection{Qualificatif de scène}

L'environnement d'une scène donnée peut également offrir aux joueurs des éléments qu'ils pourront utiliser, ou au contraire subir. Ces éléments sont représentés par les "qualificatifs de scènes". C'est au maître du jeu qu'incombe la lourde tâche d'informer les joueurs sur les possibilités de la scène. Bien sûr les joueurs sont encouragés à faire des propositions au MJ en fonction de sa description ! Un qualificatif peut à le fois être utilisée comme faiblesse, et comme trait.

Prenons quelques exemples :

\begin{itemize}
    \item L’obscurité peut être à la fois un handicap (pour viser par exemple) et un trait (pour être discret). 
    \item Une position dominante sur un champ de bataille est un trait en ce qui concerne la visée, ou la direction des troupes.
\end{itemize} \smallskip

\section{Blessures, handicap, et mort}

Dans les films (oui encore), il n'y a pas de jauge de santé au-dessus de la tête du personnage. On peut voir ses blessures, on voit que chaque grosse blessure le handicape et on déduit donc s'il est encore en état ou pas d'agir, s'il est proche ou pas de la mort.

Pour refléter ceci, nous allons utiliser un niveau de santé. Ce niveau représente l'état globale du personnage sans détailler si il s'agit d'une ou plusieurs blessures. Le type des handicaps reste donc purement narratif.

\subsection{Subir des handicaps}

Un personnage peut subir un handicap lorsqu'il perd une opposition durant un conflit. Subir un handicap signifie que le niveau de santé du personnage va diminuer d'un ou plusieurs niveaux.

Le nombre de niveau perdu par le personnage dépend de la marge d'échec du personnage : un niveau perdu pour 3 points de marge. Ainsi, un personnage qui perd une opposition d'1 point perdra uniquement un niveau. Si il la perd de 4 points, il perdra un second niveau.

Les niveaux de santé sont les suivants :

\begin{itemize}
\item Indemne
\item Égratigné (pas de malus)
\item Blessé (-1 échelons au dés)
\item Gravement blessé (-1 échelons au dés)
\item Mortellement blessé (-2 échelons au dés)
\item KO

Le joueur commence bien sur indemne. A chaque niveau de santé correspond un malus indiqué dans la liste ci-dessus. Ce malus est exprimé en échelons sur tous les dés lancés.

Par exemple, un personnage qui doit lancer D8 et D6 mais qui est blessé effectuera un jet avec D6 et D4, si il est mortellement blessé il lancera uniquement un D4.

Quand le joueur arrive à KO, c'est au maître du jeu de décider les effets exactes : inconscience, mort si jamais personne ne vient le soigner vite, mort sur le coup, ...

\end{itemize}

\section{Et si les ennemis sont nombreux ?}

Le système de conflit permet de gérer une simple escarmouche mais doit également être utilisable pour un groupe de plus grande importance ou même un combat de masse.

Dans cette situation, les petits groupes sont gérés par un dé bonus supplémentaire dépendant de l'avantage numérique du groupe sur le ou les personnage(s) et de la cohérence de ce groupe. Le dé se détermine en fonction de ces deux critères. Le dé de base est déterminé par l'avantage numérique et sera affecté par la cohésion de ce groupe, en bien ou en mal.

Voici le niveau du dé en fonction de l'avantage numérique du groupe sur le(s) personnage(s) :

\begin{itemize}
\item Léger avantage : d4.
\item Avantage moyen : d6.
\item Avantage important : d8.
\item Avantage écrasant : d10.
\end{itemize}

On prend en compte ensuite la cohésion du groupe :

\begin{itemize}
\item Groupe pas du tout coordonné, pas d'habitudes de fonctionnement en groupe : Le dé bonus baisse d'un échelon.
\item Groupe à cohésion classique, quelques belles ententes mais pas d'entraînement optimisant tout ça : On ne change pas le dé.
\item Groupe à forte cohésion, chaque membre du groupe à l'habitude de fonctionner dans le groupe : Le dé bonus augmente d'un échelon.
\end{itemize}

Cette gestion de groupe ne s'applique pas seulement à des conflits physiques mais à tous types de conflits, que cela puisse être un conflit social, mental ou d'influence...

Le niveau de santé existe toujours mais il représente l'état global du groupe et non celui d'un personnage seul.

\note{Et pour les groupes encore plus grands ?}{

    Et bien, continuez à gérer cela comme des groupes de personnages !

    Après tout, l'échelle change, mais soit il s'agira d'un conflit "classique", soit l'un des camps n'a aucune chance et l'autre le surpasse complètement. Et à ce moment là, on peut gérer cela comme un conflit contre des figurants, mais à grande échelle. 

    Bref, il n'y a pas besoin d'une gestion différente des conflits !

}

\section{Conflits sociaux}

Imaginons que le joueur soit un politicien qui a eu plusieurs aperçus de l'existence d'une vie extraterrestre sur Terre. Il doit convaincre plus haut que lui de la menace, mais n'a que peu de preuves directes en dehors du fait d'avoir été témoin. Il parvient, en faisant jouer ses contacts, à obtenir une audience auprès du chef de l'état de son pays.

Son but est donc de convaincre le chef d'état d'agir et de prendre les devants sur cette menace. 

Dans les jeux de rôle "classiques", beaucoup de MJ vont jouer la scène au roleplay, et faire un petit jet à la fin de la scène pour voir si finalement le RP sert à quelque chose.

Et bien ici, non. Comme toute action importante dans Fusina, le résultat des dés (réussite ou échec) doit être décrit par le joueur. Ici, le MJ va indiquer si le joueur réussit ou échoue, partiellement ou complètement comme dans toute action normale.

Aux joueurs et au MJ d'arriver au consensus. Pour illuster, quelques exemples appliqués à la situation décrite ci-dessus :

\begin{itemize}
\item Échec total : Le chef d'état ne croira pas du tout le politicien, et va sûrement agir pour que ce politicien perde en crédibilité.
\item Échec partiel : Le chef d'état ne croira pas du tout le politicien, et va sûrement lui demander de revenir quand il aura des preuves.
\item Réussite partielle : Le chef d'état aura des doutes, et donner des ressources au politicien pendant une durée limitée, pour obtenir des preuves avant d'agir plus ouvertement.
\item Réussite totale : Le chef d'état croit complètement le politicien, et lancera toutes les procédures qu'il faut pour annihiler la menace.
\end{itemize}

Ensuite, le joueur explique l'attitude qu'aura le chef d'état au fur et à mesure du dialogue au MJ.

Enfin, les intervenants jouent la scène, en une seule fois, sans jets de dés au milieu, dont l'issue est celle décidée plus tôt avec le MJ.

\section{Créatures et Adversité}

Le rôle de maître du jeu est de décrire et de faire vivre l'univers, mais également les personnages qui le compose. Mais comment procéder quand le maître du jeu a besoin d'évaluer la compétence d'un personnage non joueur ? Comment procéder quand les joueurs doivent faire un jet en opposition ? Faut-il définir intégralement le personnage ? Pour la plupart des cas, la réponse est non ! 

Nous allons proposer ci-dessous plusieurs méthodes dépendant principalement du type de personnage.

\subsection{Les Figurants}

Les figurants sont des personnages qu'on croise dans la partie, mais qui n'ont aucun rôle réel à jouer. Les figurants ne représentent même pas un défi pour le personnage, ils sont juste là pour faire joli, être des pots de fleurs ! Par exemple, l'herboriste qui vend ses herbes et onguents sur le marché est un figurant !

Que se passe-il si un joueur rentre en opposition avec un figurant ? La réponse est simple, il n'y a pas d'opposition ! Le figurant est un élement du décor et doit être traité comme tel. Le joueur va donc réaliser son jet contre une difficulté fixée par le maître du jeu, exactement comme s'il tentait juste d'escalader un mur.

\subsection{Les sbires}

Les sbires sont un peu plus que des figurants. Ils n'ont guère d'importance dans le déroulement du scénario mais sont tout de même là pour représenter un défi à relever, mineur certes, mais un challenge tout de même. Les sbires ont en général besoin d'être en nombre pour représenter un danger.

Comment connaître leurs caractéristiques ou leurs compétences ? Pour un sbire, le maître du jeu va uniquement définir des caractéristiques, et une occupation. 

Les caractéristiques seront définies en répartissant 4 echelons dans les caractéristiques. Exceptionnellement le maître du jeu pourra glisser 1 échelon de dé d'une caractéristique vers une autre pour spécialiser davantage le sbire. 

L'occupation correspond à l'activité principale du personnage. Il s'agit en général de son métier. Lors d'une action, l'occupation permettra de déterminer le dé utilisé pour la compétence. 

\begin{itemize}
\item Si le jet correspond directement à l'occupation du sbire, alors la compétence vaut D8.
\item Si le jet est seulement en rapport avec son occupation, alors la compétence vaut D6.
\item Dans tous les autres cas, on considère que le sbire n'a pas de compétence associée.
\end{itemize}

L'équipement est pris en compte normalement. Toutefois, en général, tous les sbires d'un même groupe portent un équipement similaire (de même catégorie) et partagent donc les mêmes bonus. Les sbires sont fait pour être gérés en groupe à l'aide des règles précisées plus haut.

Concernant la dépense de Stress pour le maître du jeu, il est possible de dépenser uniquement 1 point pour un jet concernant l'occupation du sbire. Dans tous les autres cas, pas de stress pour les sbires ! 

Prenons un exemple : des brigands. Mes brigands vont avoir les valeurs suivantes : Forcoier : D8, Esperit : D6, Entregent : D6, Sapience : D4, Influence : D4. Quand il s'agit de tendre une embuscade et de faire un peu de combat, mes brigands auront D8 en compétence. Pour fuir à travers la forêt, ils auraient un petit D6. Pour déjouer un piège mécanique, là, ils devront se contenter de leurs caractéristiques.

\subsection{Les Lieutenants}

Un lieutenant est le chef d'un groupe de sbires. C'est lui qui commande et organise sa petite troupe et la mène à l'action. Le lieutenant se génère et s'utilise de la même manière qu'un sbire à deux particularités près :

\begin{itemize}
\item Un lieutenant peut utiliser 2 points de Stress. Il peut les dépenser dans les jets directements liés à son occupation, mais aussi dans les jets en rapport.
\item Quand un groupe de sbires perd son lieutenant, il est à la déroute et travaille moins efficacement. Le dé de groupe perd donc un échelon.
\end{itemize}

\subsection{Les PNJ principaux}

Les PNJ principaux sont les personnages qui ont un vrai rôle à jouer dans l'aventure et qui sont l'égal, ou presque, des personnages. Ce sont les grands méchants, mais aussi les seconds rôles permettant de générer une ou plusieurs scènes intéressantes.

Pour les PNJ principaux il faut définir un niveau de tension. En général le niveau de tension vaut 2, voir 3 pour les PNJ vraiment dangereux ou importants. 

Pour la génération, le PNJ disposera donc de 4 + Tension échelons à répartir pour ses caractéristiques. On lui attribuera ensuite 2 occupations. L'occupation principale sera à d10, tandis que la deuxième sera à d8. Cela fonctionne ensuite comme pour les sbires et les lieutenants, les jets directement en rapport se font avec le dé de la compétence, les jets en rapport avec le dé de l'échelon en dessous, et on ne prend pas le dé de la compétence si l'action n'a rien à voir avec elle.

\subsection{Les Bêtes}

Par bêtes on entend toutes les créatures non-intelligentes et non-pensantes, que ce soit des animaux, des monstres, ou des créatures de légendes. Pourquoi faire une catégorie particulière pour les créatures ? Tout d'abord, les caractéristiques définies pour les personnages ne sont pas adéquates pour un animal. L'influence a-t-elle le moindre sens pour une bête ? De plus, les bêtes ne peuvent pas utiliser d'armes, il faut donc leur donner d'autres moyens de rester un défi.

Encore une fois nous parlons des bêtes censées présenter un réel défi : les autres bêtes seront gérées comme des figurants. 

Pour les bêtes nous allons donc définir trois caractéristiques :

\begin{itemize}
\item Forcoier : concerne toutes les actions nécessitant de la force, de l'endurance, de l'agilité, ou de la résistance.
\item Instinct : capacité à faire confiance à ses instincts, à ses perceptions.
\item Tenacité : capacité à lutter contre ses instincts, ses peurs, ses phobies. 
\end{itemize}

Leur génération est ensuite identique aux sbires, lieutenants, ou PNJs principaux.

\note{Je veux un PNJ plus puissant !}{

	\begin{quotation}
Vous désirez créer un PNJ plus costaud ? 
	\end{quotation}
	\begin{quotation}
Nous vous proposons cette solution qui donnera à votre PNJ un plus large panel de possibilités, tout en le rendant plus intéressant. Vous pouvez ajouter à ce PNJ un champ de compétence supplémentaire, qui sera considéré comme une seconde occupation. En échange, vous devez lui ajouter une faiblesse qui pourra être exploitée par les joueurs. La valeur de la nouvelle occupation sera égale à la valeur des occupations principales, diminuée d'un échelon.
	\end{quotation}
	\begin{quotation}
Vous pouvez également augmenter d'un échelon une occupation existante. Le coût est identique, vous devrez rajouter une nouvelle faiblesse à votre création.
	\end{quotation}
	\begin{quotation}
Par exemple, je décide de faire un loup boosté aux hormones. De base le loup à un champ de compétence "Chasse" à d8. Je décide de lui rajouter "Affoler les animaux tels que les chevaux" en champ de compétence, prévoyant de faire chuter les cavaliers. Je peux donc la mettre à d6 (l'échelon en dessous de d8), en échange j'impose à mon loup une faiblesse "Phobie du feu".
	\end{quotation}

}

\note{Génération partielle}{

Les règles ci-dessus permettent de générer des personnages non-joueurs en quelques minutes (sauf PNJ principaux). Mais avez-vous vraiment besoin d'une définition aussi précise ? Souvent une seule caractéristique vous sera utile pour les sbires !

Au final, nous vous conseillons de générer vos sbires et lieutenants à la volée ! Décidez des valeurs de chaque caractéristique ou équipement au moment où vous en avez besoin, pas avant. Dans mon exemple de brigand, au début j'aurai défini uniquement le Forcoier. Quand un joueur tente d'intimider, là je me pose la question du score en Âme (d6 ou d4 ?). Avant cela, pourquoi s'embêter ?

}
