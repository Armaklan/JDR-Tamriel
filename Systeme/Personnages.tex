\chapter{Introduction}

Tamriel utilise le sytsème de jeu Fusina, système qui se veut Simple, Fun, et Narratif.

Simple car nous pensons que le système ne doit pas parasiter l'ambiance de la table. L'histoire et les scènes doivent rester fluides. A aucun moment le maître du jeu ne doit avoir besoin de plonger dans son livre de règle. Le système doit donc être facilement mémorisable pour tout le monde, maître comme joueur.

Fun car nous voulons que les joueurs prennent plaisir à utiliser Fusina. La composante ludique est essentielle dans le jeu de rôle, il est donc important que les joueurs aient des mécaniques sur lesquelles influer, avec lesquelles s'amuser.

Et Narratif car nous pensons que, malgré tout, le plus important reste les personnages et l'histoire. Fusina est conçu pour aider les joueurs à vivre des aventures excitantes. Le système met en avant les personnages et leurs spécificités, ce qui les rend uniques et intéressants.

Fusina est un système en license Creative Commons. Vous avez le droit de l'utiliser, de le reproduire, de le modifier et même de commercialiser un univers l'utilisant. La seule contrainte : indiquer où trouver le système d'origine.

Je vous souhaite maintenant une bonne lecture et un bon jeu !

\section{Les principes du Système}

Ce chapitre présente les principaux élements du système de jeu, vous donnant ainsi un aperçu de ce qui vous attend. Leurs utilisations seront détaillées par la suite.

\subsection{Les dés}

Fusina utilise différents types de dés très connus des rolistes : à 4 faces, à 6 faces, à 8 faces, à 10 faces et à 12 faces. Dans le livre nous utiliserons la notion suivante "Dx" ou x représente le nombre de face du dé à utiliser. 

Nous utiliserons également une autre notation faisant intervenir plusieurs dés, comme par exemple : D4 + D8. Ce type de notation signifie qu'il faut lancer deux dés, ici un dé à 4 faces et un dé à 8 faces. Vous devrez ensuite conserver le meilleur des résultats (et non la somme).

Nous appellerons "Echelon" le passage d'un dé vers le dé supérieur. Par exemple, la différence entre un D4 et un D6 est d'un échelon. La différence entre un D8 et un D12 est de deux échelons.

\subsection{Les caractéristiques}

Les caractéristiques représentent les capacités naturelles du personnages. Est-il fort, est-il intelligent ? Ce sont les caractéristiques qui répondront à ces questions.

\begin{itemize}
\item Forcer : représente les capacités physiques (force, endurance, agilité) du personnage
\item Intelligence : représente les capacités de mémoire et de raisonnement du personnage
\item Social : représente les capacités de communication et le charisme du personnage
\item Esprit : représente la force d'âme, la volonté et la chance du personnage
\item Influence : représente le niveau social du personnage, ses moyens financiers, ses contacts, ...
\end{itemize}

Les caractéristiques ont une valeur qui va en général de D4 à D12. 

\subsection{Les compétences}

Les compétences représentent ce que le personnage a appris à faire durant sa vie. Quels enseignements a-t-il suivis ? Quelles professions a-t-il pratiquées ?

Les compétences ont une valeur qui va de D4 à D12.

\subsection{Les équipements}

L'équipement d'un personnage représente ses possessions, les objets qu'il possède. Dans le système Fusina, la plupart des équipements n'ont pas d'effet réèl, ce sont des artifices purement narratifs. 

\subsection{Les traits}

Les traits sont des adjectifs ou des phrases qui caractérisent le personnage. Ils peuvent définir son apparence, son histoire, son caractère. Tous ce qui rend le personnage particulier, ce qui le fait sortir du commun des mortels, est un trait. Les traits ne sont pas exclusivement les aspects positifs de votre personnage, le négatif fait aussi parti de son charme !

En jeu, les traits peuvent être utilisé comme bonus sur certaines de vos actions. Vous pourrez également choisir de vous pénaliser en mettant en avant l'aspect négatif de l'un de vos traits. Les traits vous seront utiles pour rendre votre personnage unique et le mettre en scène autant dans ses bons que ses sombres cotés.

\subsection{Résolution d'une action}

Le système de jeu intervient quand un personnage tente une action qu'il n'est pas sûr de réussir ou d'échouer. La résolution d'une action prend en compte les caractéristiques et les compétences. Le joueur lancera les dés correspondant à ces deux élements. Les traits pourront ensuite intervenir comme bonus pour l'aider à accomplir des exploits plus importants.


\chapter{Création de personnage}

\section{Concept}

Premier étape et surement la plus importante de la création : définir le concept de votre personnage. Que fait-il ? Qu'est-ce qui le motive ? A quoi il ressemble ? Peut-être avez vous en tête un personnage de film ou de roman qui vous inspire ? Et surtout à Tamriel: à quel peuple appartient-il ?

Si vous n'avez pas d'idée, prenez le temps d'en discuter avec votre maître du jeu. Il vous sera impossible de continuer la création sans avoir au moins une vague idée de ce que vous voulez créer.

\section{Les Traits}

Votre personnage est un être unique avec ses forces et ses faiblesses, ses particularités, ses motivations et même son histoire propre. Même si d'autres auront surement les mêmes caractéstiques ou compétences, ils seront bel et bien différents. Les traits servent à représenter cette différence, ce qui vous rend hors du commun.

Lors de la création du personnage, vous devez commencer par choisir ses traits. Un personnage standard en possède six. Toutefois, selon l'univers et l'ambiance qu'il vise, le maître du jeu pourra vous en demander un nombre différent.

\exemple{Quelques traits}{

\begin{itemize}
\item Forte tête
\item Montagne de muscle
\item Agilité féline
\item Membre de la haute société
\item Bas du front !
\item Peur du vide.
\item Incapable de communiquer avec la gente féminine.
\item Exilé.
\end{itemize}

}

Faites attention à ne pas choisir que des traits fondamentalements positifs. Les mauvais cotés de votre personnage auront également leurs importances une fois en jeu. N'hésitez donc pas à prendre des traits négatifs, ou des traits ayant à la fois leurs avantages et défauts.

\section{Les Caractéristiques}

Les caractéristiques représentent les capacités naturels du personnages. Est-il fort, est-il intelligent ? Ce sont les caractéristiques qui répondront à ces questions.

Les caractéristiques ont une valeur qui va de d4 à d12. Elle est évaluée sur cette échelle :

\begin{itemize}
\item D4 : Faible
\item D6 : Standard
\item D8 : Supérieur à la moyenne
\item D10 : Excellent
\item D12 : Héroïque
\end{itemize}

Pour déterminer le score d'une caractéristique, nous allons définir, pour chaque trait, une tendance. Elle représente la caractéristique qui semble le plus proche de l'utilisation de ce trait. Le score d'une caractéristique dépend du nombre de traits associés.

\begin{itemize}
\item 0 trait : D4
\item 1 trait : D6
\item 2 traits : D8
\item 3 traits : D10
\item 4 traits : D12
\end{itemize}

Par exemple, "Forte tête" indique une forte volonté du personnage, la tendance de ce trait est donc l'âme. "Montagne de muscle" serait plutôt physique, ...

Il peut arriver que certains traits n'aient pas de tendance : soit ils sont tout simplement négatif, soit vous ne voyez pas à quoi le rattacher. Dans ce cas, vous pouvez choisir librement la caractéristique qui pourra augmenter d'un échelon. Cette possibilité doit bien sur être exploitée en accord avec le maitre du jeu.

\section{Les Compétences}

Les compétences représentent ce que le personnage a appris à faire durant sa vie. Quels enseignements a-t-il suivis ? Quelles professions a-t-il pratiquées ? 

Les compétences ont une valeur qui va de d4 à d12 selon l'échelle suivante : 

\begin{itemize}
\item d4 : Vagues connaissances
\item d6 : Amateur
\item d8 : Professionnel
\item d10 : Expert
\item d12 : Grand Maître
\end{itemize}

Le joueur dispose de 25 points à dépenser pour inscrire des compétences sur sa feuille. Les compétences sont totalement libres. Le niveau de la compétence dépend du nombre de points investis :

\begin{itemize}
\item D4 : 1 points
\item D6 : 3 points
\item D8 : 6 points
\item D10 : 10 points
\item D12 : 15 points
\end{itemize}

\exemple{Exemples de compétences}{

\begin{itemize}
        \item Comédien
        \item Forgeon 
        \item Séduire la donzelle
        \item Connaissance de l4empire
        \item Évoluer dans la haute noblesse
        \item Maître d’arme
\end{itemize}

}

\section{La Magie}

La magie dans Tamriel est liée à l'énergie présente dans chaque être, la magicka.

Elle est instinctive, mais liée à de nombreux facteurs. En effet, selon le type d'effet recherché, non seulement la magie est différente, mais en plus cela ne se ressent pas de la même manière. Ainsi, chaque mage développe des compétences différentes pour chaque domaine de magie. Il est d'ailleurs difficile d'apprendre à faire des sorts utiles sans prendre le temps ou être guidé.

Les domaines existants sont :

\begin{itemize}
\item
L'altération : les effets de ce type de magie modifient réellement les choses, voir les déforment. Les modifications sont permanentes pour les choses non vivantes, et temporaires pour les choses vivantes.
\item
La conjuration : ce domaine regroupe tout ce qui concerne l'invocation et le bannissement d'êtres provenant ou non de Tamriel.
\item
La destruction : ici on affaiblit, on blesse, on détruit. C'est le domaine de magie le plus offensif, et cela se comprend !
\item
L'illusion : comme son nom l'indique, ce domaine regroupe tout ce qui est lié à la supercherie, à la tromperie, au faussage des sens.
\item
La nécromancie : ce domaine regroupe tout ce qui concerne la modification de l'état d'un être autrefois vivant. C'est dans ce domaine magique que l'on retrouve également les malédictions et les maladies non naturelles. Ce domaine est très mal vu par toutes les autorités.
\item
Le mysticisme : ce domaine regroupe tous les effets transcendant ou supprimant les limites de la réalité. Voler, se téléporter, voilà qui est pratique bien que dangereux !
\item
La restauration : Réparer, restaurer, protéger, tout cela est du domaine de la restauration.
\end{itemize}

Les domaines se notent dans les compétences, et s'achètent au même tarif que les compétences, mais avec un pool de points dépendant du score d'Intellect selon ce tableau :
\\
\\
\begin{tabular}{|c|c|}
    \hline
    Score d'Intellect & Nombre de points à dépenser \\ \hline \hline
    d4 & 0  \\ \hline
    d6 & 2 \\ \hline
    d8 & 6  \\ \hline
    d10 & 12 \\ \hline
    d12 & 20 \\ \hline
\end{tabular}

Cependant, on ne pourra atteindre dans un domaine magique le niveau d'Intellect. Ce qui signifie qu'un personnage à d4 en Intellect ne pratiquera probablement jamais la magie.

\exemple{Prenons un exemple}
{Comme Jiub a d10 en Intellect, il a 12 points à dépenser pour l'achat de compétences dans les domaines magiques, et il ne pourra atteindre dans un domaine la valeur de d10, mais il pourra avoir un domaine à d8, car cela reste sous sa valeur d'Intellect.}

\section{L'équipement}

\subsection{Généralités}

Le joueur constitue ensuite une liste d'équipement qu'il désire avoir et qu'il est logique que son personnage ait.

Pour chaque équipement, le MJ a 3 solutions : 

\begin{itemize}
\item Accepter l'équipement : l'équipement est logique pour le personnage et ne requiert pas d'être particulièrement riche. 
\item Refuser l'équipement : au contraire, l'équipement peut être totalement illogique pour le personnage (équipement que le personnage ne sait pas utiliser), ou coûter excessivement cher.
\item Demander un jet d'Influence : l'influence est utilisée pour simuler les contacts du personnages, ainsi que ses ressources. Le jet d'influence permet de savoir si le personnage a pu mettre la main sur l'équipement en question, et s'il a dû s'endetter pour le faire.
\end{itemize}

Le MJ peut utiliser l’une ou l’autre de ces conditions, à sa propre convenance. Nous incitons toutefois à utiliser le jet d’influence pour tous les cas litigieux.

En termes de résolution d’action, l’équipement est purement narratif. Il sert à mettre de la couleur dans les descriptions, et peut permettre d’effectuer des actions qu’il aurait été impossible de faire autrement. L’équipement n’apporte donc aucun bonus. Nous jugeons que le plus important reste le personnage et ses capacités propres, pas ce qu’il porte !

Concernant les équipements spéciaux (objet magique par exemple), ils peuvent être gérés aisément à l’aide de traits supplémentaires.

\exemple{Quelques équipements...}{

\begin{itemize}
\item Dague
\item Matériel d'escalade
\item Tenue de camouflage
\item Vieux cheval de famille
\end{itemize}

}

\subsection{Armures et Boucliers}

Dans Tamriel, certains combattants, selon leur métier et leur statut (qui se mesure avec la caractéristique Influence, rappelons-le) peuvent avoir des boucliers et/ou des armures.

Les armures/protections fonctionnent de manière identique au système de base de Fusina, avec une valeur fixe indiquant des niveaux de blessures diminuables sur une scène. Par exemple, une armure de protection 2, peut au choix transformer deux blessures en deux blessures d'un niveau moins grave (par exemple deux mortelles en deux graves), ou alors baisser de 2 niveaux une blessure (une mortelle devient une blessure légère, par exemple).

Le bouclier, lui, fonctionne comme l'armure (il peut absorber des blessures) avec un niveau de 1 à 3, sauf que l'on peut s'en servir plusieurs fois par scène, à la seule et unique condition d'indiquer défendre/tenir une position !