\documentclass{Tamriel}

\begin{document}

\chapter*{Eliel Grimhe - Altmer cartographe}

\begin{multicols*}{2}
\raggedcolumns

        \section*{Description}

        Eliel est un altmer bien plus cultivé que ses pairs, et qui en paie le prix : toujours fourré dans ses livres, il en paierait le prix social s'il n'était également un linguiste des plus demandés. Il ne se contente pas d'être un puit de connaissances sur Tamriel, il a aussi fait des études intensives sur le continent d'Akavir, qu'il sait être composé en réalité d'un archipel d'îles plus une assez importante île principale, plus grande de Cyrodiil, c'est pour dire ! Il a pratiqué les langues de toutes les espèces connues là-bas par les quelques Tsaesci encore dans l'Empire, à savoir le Tsaesci évidemment mais aussi l'Imga, la langue de leurs anciens esclaves. 
        
        Eliel est calme et posé. Il prend le temps de réfléchir avant d'agir, et y réfléchit à plusieurs fois avant de tenter quelque chose de dangereux. Certains prétendent qu'il s'agit de peur, lui-même préfère parler de prudence. Toujours est-il qu'il ne fonce pas dans le tas. Son appréciation de la magie s'en ressent, il ne pratique que la Restauration et le Mysticisme (qui lui permet de défier le temps et de fuir plus rapidement).
        
        La dernière demande qu'on lui a faite l'angoisse et l'emplit de joie en même temps : on souhaite qu'il embarque avec de futurs colons sur Akavir, car les autorités souhaitent envoyer des hommes en avant d'une armée. Et pour qu'une armée puisse débarquer, il faut tâter et préparer le terrain... Il n'ira pas seul pour cela, et on l'a assuré d'un paiement plus qu'adéquat, en plus de la cession d'un esclave argonien dont les qualités d'éclaireur et d'informateur pourraient aider : Teinaava.
        
        \columnbreak

        \section*{Notes}
        
\end{multicols*}

\chapter*{Eliel Grimhe - Altmer cartographe}

\begin{multicols*}{2}
\raggedcolumns   

        \section*{Traits}

        \begin{itemize}
        \item Meilleur cartographe selon lui-même (Intellect)
        \item (altmer) C'est un puits de connaissance (Intellect)
        \item On ne dérange pas un érudit en plein travail (Âme)
        \item Peu de gens ont mes connaissances, et beaucoup feraient tout pour elles ! (Influence)
        \item (altmer) Extrêmement arrogant. (Âme)
        \item Sa "prudence" le guide du côté opposé au danger.  (Intellect)
        \end{itemize}
        
        \section*{Équipement}

        \begin{itemize}
        \item 
        \item 
        \item 
        \end{itemize}
        
        \section*{Blessures}

	\begin{itemize}
	\item Indemne
	\item Égratigné (pas de malus)
	\item Blessé (-1 échelons au dés)
	\item Gravement blessé (-1 échelons au dés)
	\item Mortellement blessé (-2 échelons au dés)
        \end{itemize}
        
        \columnbreak
                
        \section*{Caractéristiques}

        \begin{itemize}
        \item Physique : D4
        \item Intellect : D10
        \item Social : D4
        \item Ame : D8
        \item Influence : D6
        \end{itemize}

        \section*{Compétences}

        \begin{itemize}
        \item Tout savoir : D10
        \item Parler des langues que personne ne connaît : D8
        \item Diplomatie : D6
        \item S'est entraîné à fuir des livres dans les mains : D6
        \item Un altmer invisible ? Ça existe ! : D6
        \end{itemize}
        
        \section*{Magie}

        \begin{itemize}
        \item Restauration : D8
        \item Mysticisme : D8
        \end{itemize}

\end{multicols*}

\chapter*{Teinaava - Argonien éclaireur/informateur}

\begin{multicols*}{2}
\raggedcolumns

        \section*{Description}

        Bon, déjà, être esclave, c'est pas une condition sympathique quand on reste toujours au même endroit. Mais alors quand ses maîtres l'envoient régulièrement voir ce qui se passe dans les endroits mal famés et où les maladies pullulent, c'est vraiment pas chouette. Qu'il soit argonien et donc très résistant aux maladies doit y être pour quelque chose, c'est certain.
        
        Teinaava est grand, agile, pas spécialement balaise mais il a un doigté et des réflexes que n'importe qui rêverait d'avoir (même d'autres argoniens). Il est aussi finalement très peu différent des autres esclaves. C'est la force des argoniens : comme leurs mâchoires ne sont pas faites pour prononcer le Tamriélique, tout le monde les prendre pour des demeurés. Mais Teinaava n'est pas du tout stupide. Il écoute, il apprend vite, et utilise les préjugés à son avantage. Personne ne se méfie de l'argonien ? Dommage...
        
        On l'a confié à un altmer, Eliel. Il l'aime bien, à première vue, bien qu'il soit du genre à rester dans un bureau, ce n'est qu'un prétentieux pas dangereux, qui est cependant très gentil derrière son "masque" altmer. Reste que Teinaava doit le protéger, et cela passera par l'étape de renseignements. Pour réfléchir correctement, l'Intellectuel a besoin des bonnes infos, et d'être vivant. C'est ça que Teinaava doit faire... pour l'instant. Mais sur une île loin de tout, si les conditions et les alliés se trouvent, pourquoi ne pas faire ce qu'il souhaite depuis un moment : être libre ?
        
        \columnbreak

        \section*{Notes}
        
\end{multicols*}

\chapter*{Teinaava - Argonien éclaireur/informateur}

\begin{multicols*}{2}
\raggedcolumns
    
        \section*{Traits}

        \begin{itemize}
        \item (argonien) Insensible aux toxines et maladies (Physique)
        \item Les argoniens entendent très très bien (Social)
        \item Il a des réflexes incroyables (Physique)
        \item Un argonien, surtout celui-là, c'est tout sauf stupide (Intellect)
        \item (argonien) La seule chose qu'il souhaite par dessus tout : la liberté. (Ame)
        \item Mal vu en dehors des milieux louches... (Influence)
        \end{itemize}
        
        \section*{Équipement}

        \begin{itemize}
        \item 
        \item 
        \item 
        \end{itemize}
        
        \section*{Blessures}

	\begin{itemize}
	\item Indemne
	\item Égratigné (pas de malus)
	\item Blessé (-1 échelons au dés)
	\item Gravement blessé (-1 échelons au dés)
	\item Mortellement blessé (-2 échelons au dés)
        \end{itemize}
        
        \columnbreak
        
        \section*{Caractéristiques}

        \begin{itemize}
        \item Physique : D8
        \item Intellect : D6
        \item Social : D6
        \item Ame : D6
        \item Influence : D6
        \end{itemize}

        \section*{Compétences}

        \begin{itemize}
        \item Je te sussure à l'oreille, tu n'y vois que du feu : D8
        \item Malandrin, espion, on m'a donné beaucoup de titres : D8
        \item Quand tu pars devant il faut savoir revenir : D8
        \item Si tu veux me surprendre, il va falloir te lever tôt : D8
        \item Étiquette et manières, on l'a forcé à les apprendre : D4
        \end{itemize}
        
        \section*{Magie}

        \begin{itemize}
        \item Illusion : D6
        \item Mysticisme : D6
        \end{itemize}

\end{multicols*}

\chapter*{Ymir Trijdal - Nordique capitaine de son navire}

\begin{multicols*}{2}
\raggedcolumns

        \section*{Description}

        Un nordique marin, Ymir est la preuve que cela existe !
        
        Abandonné tout petit, il a d'abord été mousse, puis devant sa force, sa capacité à se faire respecter et à commander, il a fini par grimper les échelons ! Alors forcément, ça fait des jaloux, mais Ymir il pète la gueule à tous les mécontents et sa vie reste simple. Il ne faut d'ailleurs pas lui demander la lune, parce que ce qu'il sait faire, c'est motiver son équipage et aborder des bateaux pirates, c'est d'ailleurs un de ses loisirs favoris !
        
        La mer et lui sont amoureux depuis longtemps, et il navigue dans des eaux dangereuses, pour le frisson de l'aventure ! Son physique massif, titanesque presque, ses cheveux blonds, sa peau mate, les muscles saillants : on y réfléchit à deux fois avant de le chercher ! Il sait naviguer comme personne, il est respecté et/ou craint, ainsi, quand on lui proposa une somme réellement impressionnante pour amener deux bateaux de colons en Akavir, le défi lui sembla plus qu'intéressant...
        
        \columnbreak

        \section*{Notes}
        
\end{multicols*}

\chapter*{Ymir Trijdal - Nordique capitaine de son navire}

\begin{multicols*}{2}
\raggedcolumns

        \section*{Traits}

        \begin{itemize}
        \item On me surnomme le titan des mers (Physique)
        \item La mer, c'est ma terre. Je m'y sens bien. (Âme)
        \item Quand t'es sur le pont, c'est un combat de tous les instants (Physique)
        \item Quand tu navigues, tu prends du poids. En muscles. (Physique) 
        \item Se repérer aux étoiles, ça c'est utile (Intellect)
        \item (nordique) Incapable de fuir un combat (Ame)
        \end{itemize}
        
        \section*{Équipement}

        \begin{itemize}
        \item 
        \item 
        \item 
        \end{itemize}
        
        \section*{Blessures}

	\begin{itemize}
	\item Indemne
	\item Égratigné (pas de malus)
	\item Blessé (-1 échelons au dés)
	\item Gravement blessé (-1 échelons au dés)
	\item Mortellement blessé (-2 échelons au dés)
        \end{itemize}
        
        \columnbreak
        
        \section*{Caractéristiques}

        \begin{itemize}
        \item Physique : D10
        \item Intellect : D6
        \item Social : D4
        \item Ame : D8
        \item Influence : D4
        \end{itemize}

        \section*{Compétences}

        \begin{itemize}
        \item La vie de marin, je gère : D10
        \item Je manie l'épée, il faut bien, pour tuer des pirates : D8
        \item Commander un navire. Et l'équipage qu'est dedans : D8
        \item La bagarre, c'est le quotidien, sur un bateau : D6
        \end{itemize}
        
        \section*{Magie}

        \begin{itemize}
        \item Destruction : D6
        \item Altération : D6
        \end{itemize}
        

\end{multicols*}

\chapter*{Ks'aashra - Khajiit chasseuse}

\begin{multicols*}{2}
\raggedcolumns

        \section*{Description}
        
        Ks'aashra est une esclave, comme Teinaava. Mais contrairement à ce dernier, elle est une Khajiit. Les Khajiits sont plus proches des animaux que des mers (impériaux, altmer, etc...). Leurs instincts, leur rage, ils ont beaucoup plus de mal à les contenir que les argoniens. Bien qu'elle soit agile, Ks'aashra est avant tout une combattante. Sa souplesse et sa vitesse ne sont qu'un atout de plus aux qualités naturelles qu'elle a : des griffes acérées, et des crocs qui jaillissent d'une machoire capable de solidement broyer ce qu'elle contient.
        
        Achetée par Ymir il y a de cela longtemps, Ks'aashra est sa brute personnelle. Sa force de frappe. Les derniers mousses qui ont contredit le capitaine ont été retrouvés tabassés. Lacérés. Il n'y avait qu'une personne à bord capable de tout ça. Ks'aashra en souriait. Elle apprécie le capitaine qui ne la considère pas réellement comme une esclave.
        
        \columnbreak

        \section*{Notes}
        
\end{multicols*}

\chapter*{Ks'aashra - Khajiit chasseuse}

\begin{multicols*}{2}
\raggedcolumns
        
        \section*{Traits}

        \begin{itemize}
        \item (khajiit) Agilité Féline (Physique)
        \item Quand elle est enragée, rien ne la retient (Physique)
        \item Renifler et sentir la proie, c'est un don (Intellect)
        \item Dans la jungle, tout n'est qu'obstacles que j'évite. (Physique)
        \item (khajiit) Dépendante au Skooma (une drogue) (Physique)
        \item Une fois un combat engagé, l'un des deux doit être tué. C'est la loi du plus fort. (Ame)
        \end{itemize}
        
        \section*{Équipement}

        \begin{itemize}
        \item 
        \item 
        \item 
        \end{itemize}
        
        \section*{Blessures}

	\begin{itemize}
	\item Indemne
	\item Égratigné (pas de malus)
	\item Blessé (-1 échelons au dés)
	\item Gravement blessé (-1 échelons au dés)
	\item Mortellement blessé (-2 échelons au dés)
        \end{itemize}
        
        \columnbreak
        
        \section*{Caractéristiques}

        \begin{itemize}
        \item Physique : D12
        \item Intellect : D6
        \item Social : D4
        \item Ame : D6
        \item Influence : D4
        \end{itemize}

        \section*{Compétences}

        \begin{itemize}
        \item Traquer sa proie : D8
        \item Tuer sa proie : D8
        \item Surprendre sa proie : D8
        \item Impressionner sa proie : D8
        \item Survivre à l'hiver : D4
        \end{itemize}
        
        \section*{Magie}

        \begin{itemize}
        \item Restauration : D6
        \item Destruction : D6
        \end{itemize}

\end{multicols*}

\chapter*{Riagrin Morden - Dunmer bourrin et charmeur}

\begin{multicols*}{2}
\raggedcolumns

        \section*{Description}
        
        L'autre capitaine, celui de l'autre gros navire de colons, c'est pas un tendre.
        
        Un dunmer comme lui, y en a pas beaucoup. Il parcourt les mers depuis un certain temps, mais bien que capitaine de son navire, c'est pas pour ça qu'il est connu, ni pour ses capacités de navigateur hors pair. C'est un corsaire, les huiles de Cyrodiil le chargent depuis des années de défendre leurs biens sur les mers, et c'est exactement ce qu'il adore faire. Son activité principale consistait à trouver et décimer les pirates sur les voies commerciales entre Morrowind et Solstheim, pour le compte de la compagnie minière de l'Est.
        
        Quand il y a peu, on lui a proposé de "conquérir" un continent, ou tout du moins d'aider à y poser les fondements d'une colonie impériale avant l'arrivée du gros des troupes, contre somme conséquente et droit à sa part sur les biens récoltés là-bas, ça lui a plu, et il a accepté rapidement.
        
        Il compte aussi suivre les directives du cartographe, Eliel, parce que ce dernier a l'air de savoir où il va, tout comme le capitaine de l'autre navire. Hmmm, peut-être qu'il aurait du apprendre lui aussi à être malin ?
        
        \columnbreak

        \section*{Notes}
        
\end{multicols*}

\chapter*{Riagrin Morden - Dunmer bourrin et charmeur}

\begin{multicols*}{2}
\raggedcolumns

        \section*{Traits}

        \begin{itemize}
        \item (dunmer) Sait reconnaître les menteurs (Social)
        \item Par la force, on peut se sortir de toutes les situations (Physique)
        \item Je vis pour combattre (Âme)
        \item Embobiner les gens c'est ma seconde nature (Social)
        \item La plume est vraiment moins forte que l'épée (Physique)
        \item (dunmer) Très mal vu en société (Social)
        \end{itemize}
        
        \section*{Équipement}

        \begin{itemize}
        \item 
        \item 
        \item 
        \end{itemize}
        
        \section*{Blessures}

	\begin{itemize}
	\item Indemne
	\item Égratigné (pas de malus)
	\item Blessé (-1 échelons au dés)
	\item Gravement blessé (-1 échelons au dés)
	\item Mortellement blessé (-2 échelons au dés)
        \end{itemize}

        \columnbreak
        
        \section*{Caractéristiques}

        \begin{itemize}
        \item Physique : D8
        \item Intellect : D4
        \item Social : D10
        \item Ame : D6
        \item Influence : D4
        \end{itemize}

        \section*{Compétences}

        \begin{itemize}
        \item Combat acrobatique : D10
        \item Tu vas m'écouter, oui ? : D8
        \item Naviguer et aborder l'ennemi : D6
        \item Repérer par où les ennuis peuvent arriver : D6
        \item Survivre en milieu hostile, j'ai déjà fait : D6
        \end{itemize}

        
\end{multicols*}
\end{document}
